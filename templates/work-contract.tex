\chapter{Work Contract}
\section*{Members}
Birgitte Thoresen, Chris Sivert Sylte, Vegard Mytting

\section*{Introduction}
This work contract is based on a set of typical goals, task distributions, procedures, and guidelines for interaction in student projects. The contract is supplemented with individual interpretations of these aspects and how they will be achieved.  
Points are added or removed as deemed necessary to tailor the contract to the specific task.  

\section*{Roles and Task Distribution}
\textbf{How is the work organized?}  

Document Manager: Birgitte is responsible for recording and documenting the key points and decisions made during group meetings.
Quality Assurance: Vegard ensures that code and other deliverables meet the required standards.
Meeting Leader: Chris Sivert is responsible for organizing and facilitating meetings, ensuring they run smoothly and efficiently.

\section*{Procedures (How are things done?)}
\begin{itemize}
    \setlength{\itemsep}{10pt}
    \item \textbf{Meetings}  
    Meetings with the supervisor and product owner are scheduled via email and formal meeting invitations. Group meetings are arranged as needed, typically when an internal update is required.

    \item \textbf{Notification of Absence or Other Incidents}  
    Notify as early as possible in case of absence. If unforeseen events occur, inform the group as soon as possible and provide necessary updates.

    \item \textbf{Document Management}  
    Documents are stored in Overleaf and OneDrive. GitHub is used for version control.

    \item \textbf{Submission of Group Work}   
    Each member is responsible for reviewing their contributions before pushing to ensure no mistakes are made. Quality control will involve thorough testing, code reviews, and verification that all project requirements and coding standards are met. During meetings, the group will review progress, adjust plans as needed, and ensure alignment with deadlines.
    
\end{itemize}

\section*{Interaction (How Do We Collaborate?)}
\begin{itemize}
    \setlength{\itemsep}{10pt}
    \item \textbf{Attendance and Preparation}  
    What is considered an acceptable meeting time for group meetings and lectures? What are the preparation requirements?

    We have regular meetings on Thursday mornings, and additional meetings are scheduled as needed. After week 11, meetings will shift from Thursdays to Mondays, as the course INGA2300 will be completed by then.

    \item \textbf{Presence and Engagement}  
    How should we address the use of PCs for entertainment during work sessions?

    Using PCs or other devices for entertainment is fine as long as the work is being completed and deadlines are met.

    \item \textbf{How to Support Each Other}  
    How can we ensure that everyone looks forward to the next workday?

    Stay positive, help each other when needed, and make sure to take regular breaks. Organize and schedule tasks effectively, and always have a clear plan for the work ahead.

    \item \textbf{Disagreements and Breaches of Agreement}  


    Since our group consists of three members, decisions will be made based on a majority vote. Each member is expected to take a side; neutrality is not an option. In case of disagreements, open communication and constructive discussion will be encouraged to reach a resolution. 

    Deviations, such as failing to notify when unable to attend or meet, being rude, or not following through on agreed tasks, will be addressed and resolved through discussion to maintain a positive and collaborative environment.
    
\end{itemize}
