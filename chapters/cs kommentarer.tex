
%---TEORI----%


%---De tre avsnittene under er fra teori, og kan muligens være i diskusjon----%


%----DISKUSJON----%

%4.3 API (diskusjon)

%and improve the responsiveness of the frontend interface

%This method was suggested by the product owner and was also approved by the supervisor. 

%This abstraction simplifies integration into the user interface and ensures that resources are properly cleaned up when the component is unmounted. Overall, the use of WebSocket communication is a key component in delivering a responsive and live experience, particularly for tournaments and public displays where move-by-move tracking is essential.

%This provides a seamless viewing experience for users following live games.

%On the frontend, WebSocket functionality is encapsulated within a custom React hook, which handles connection lifecycle management and state synchronization. jeg fjerna denne siden det er allerede forklart lifecycle og state tidligere i avsnittene



%Endra til små bokstaver i 2.7 Project Managaement


%\subsubsection*{Virtual Machine}
%\label{subsubsec:virtual-machine}

%A \gls{vm} is a software-based emulation of a physical computer. It runs on a host machine but operates with its own virtual \gls{cpu}, memory, and storage. Because the \gls{vm} is isolated from the host system, processes running inside it cannot directly affect the host. This isolation makes \glspl{vm} ideal for testing, running different operating systems, or creating separate software environments \cite{microsoft:virtual-machine}.

%Jeg fjernet vm ettersom vi brukte det kun til requirements.txt og vi ikke snakker om det senere i teksten
