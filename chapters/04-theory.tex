\chapter{Theory}
\label{chp:theory}

\begin{center}
    \textit{This chapter delves into the theoretical foundations of the technologies and methodologies employed in the project. It covers key concepts such as computer vision, image recognition, and the integration of chess analysis tools like \Gls{stockfish}.}    
\end{center}

\section{Literature Review}
\label{sec:literature-review}

As technology continues to evolve, a variety of digital tools and platforms have emerged to enhance the experience of playing and analyzing chess. Leading online platforms such as \textit{chess.com} and \textit{lichess.org} provide global matchmaking, tutorials, and advanced game analysis features, significantly transforming how chess is played and studied. \\

In parallel with these software-based innovations, physical chessboards and traditional tournament practices have also been modernized through the integration of digital technologies. For instance, the use of \gls{rfid} tags enables the digitization of over-the-board games, allowing for automated tracking and broadcasting of moves. \cite{quora:shah} This approach bridges the gap between physical gameplay and digital analysis. \\

To further streamline tournament operations, electronic scoresheet systems have been introduced as alternatives to traditional paper-based methods. One such solution is Clono, a tablet-based system that transmits game data directly to a central server. Clono facilitates live online broadcasts of tournaments while offering a secure, affordable, and easy-to-use interface that eliminates the need for electronic chessboards, cables, or on-site technical personnel. \cite{clono} \\

Advancements in \gls{ai} have also enabled the development of tools that automate the digitization of chess games through visual recognition. A prominent example is \textit{ChessCam}, a web- and mobile-based application that processes video footage or live streams to detect moves and generate \gls{pgn} files for analysis or archival. This approach makes it possible to digitize games with minimal hardware and manual input. \cite{chess:chesscam, lichess:chesscam}

\section{Machine Learning} \label{sec:machine-learning}

As a subfield of \gls{ai}, \textbf{computer vision} focuses on enabling machines to interpret, analyze, and extract meaningful information from visual data, such as digital images and videos. The primary goal of computer vision is to replicate the human ability to perceive and understand visual information, as illustrated in Figure~\ref{fig:computer-vision}

\begin{figure}[h!] \centering \includegraphics[width=0.75\linewidth]{figures/theory/machine-learning/computer-vision.png} \caption[Computer Vision vs. Human Vision]{Comparison of Computer Vision and Human Vision: Illustrating the similarities and differences in processing visual data \cite{turing:computer-vision}} \label{fig:computer-vision} \end{figure}

One of the most influential technologies in modern \gls{ml} is the \textbf{artificial neural network} (ANN). These networks excel at handling a wide range of data types, including images, audio, and text. Different neural network architectures are suited for specific tasks. For instance, recurring neural networks (RNNs), especially those incorporating \gls{lstm}, are effective for sequential data such as text. \\

In contrast, \textbf{\glspl{cnn}} are particularly effective in processing image data. Their architecture leverages convolutional layers to extract spatial features, making them highly efficient in image classification, object detection, and segmentation tasks. An example of such a network applied to handwritten digit classification is shown in Figure~\ref{fig:convolutional-neural-network}. \\

\newpage

\begin{figure}[h!] \centering \includegraphics[width=0.75\linewidth]{figures/theory/machine-learning/convolutional-neural-network.png} \caption[Example of CNN architecture for handwritten digit classification]{CNN architecture for classifying handwritten digits \cite{medium:cnn}} \label{fig:convolutional-neural-network} \end{figure}


Due to their ability to learn hierarchical representations of visual data, CNNs have become a foundational tool in many computer vision applications, ranging from medical diagnostics to autonomous driving. Another fundamental concept in machine learning is \textbf{supervised learning}, where models are trained on labeled datasets. Each example in the dataset includes both an input and the correct output, allowing the model to learn how the input relates to the expected output. After training, the model can generalize this learned relationship to make predictions on new, unseen data \cite{geeksforgeeks:supervised-learning, google:supervised-learning}. \\

Within supervised learning, \textbf{classification} refers to predicting categorical outcomes. For example, a classification model trained on images of geometric shapes can learn to associate visual features with specific shape categories. When shown a new image, the model attempts to determine the most likely category based on its prior learning. This process is illustrated in Figure~\ref{fig:supervised-learning}, which shows how labeled data is used to train a model to make accurate predictions on unseen inputs. \\

\begin{figure}[h!] \centering \includegraphics[width=0.75\linewidth]{figures/theory/machine-learning/supervised-learning.png} \caption[Supervised Learning with labeled data]{Supervised learning process using labeled data for classification \cite{tpointtech:supervised-learning}} \label{fig:supervised-learning} \end{figure}

By combining supervised learning with powerful models like CNNs, machine learning systems have achieved remarkable success in tasks that require human-like perception and decision-making [KILDE?]. As these models become more accurate and versatile, deploying them efficiently to real-world applications becomes increasingly important.  \\

To facilitate this, \textbf{\gls{onnx}} is an open-source format for representing machine learning models, enabling easy transfer and interoperability between different frameworks like PyTorch and TensorFlow \cite{roboflow:onnx}. \\

Similarly, \textbf{tensorflow} is an open-source machine learning framework developed by Google for building and deploying machine learning models. It provides comprehensive tools for deep learning, image processing, and natural language processing, and supports both training and inference of models \cite{nvidia:tensorflow}. \\

\textbf{Inference} refers to the process of using a trained machine learning model to make predictions on new, unseen data. In the context of object detection models, inference involves feeding an input image into the model, which then processes the image and outputs predictions. The predictions typically consist of object classes, confidence scores, and bounding box coordinates \cite{nvidia:inference}. \\

\section{Object Detection}

\textbf{\gls{leyolo}} is a \gls{cnn} designed for real-time object detection. It is a lightweight version of \gls{yolo} where its simplified architecture. allows it to achieve significantly faster inference times. This makes it particularly suitable for real-time applications, where quick detection is more important than achieving the highest possible accuracy. Like its predecessor, it divides an image into a grid, and each cell predicts whether an object is present, along with its bounding box coordinates \cite{openreview:leyolo}.\\

Central to object detection is the use of concept of \textbf{bounding boxes}. Bounding boxes are rectangular regions used to indicate the location of an object within an image. It is typically represented by four values: the center coordinates \((x_c, y_c)\), which define the center of the box, and the width \((w)\) and height \((h)\), which define the dimensions of the box. In object detection tasks, models predict these coordinates to both localize and classify objects within an image \cite{peopleforai:boundingbox}. \\

\newpage

\begin{figure}[h!]
    \centering
    \includegraphics[width=0.75\linewidth]{figures/theory/image-recognition/bbox-example.png}
    \caption[Example of a bounding box in object detection]{Example of a bounding box used in object detection to localize an object within an image \cite{peopleforai:boundingbox}}
    \label{fig:boundingbox}
\end{figure}

To improve detection accuracy across different object sizes and shapes, models use \textbf{Anchor boxes}. Anchor boxes are predefined bounding boxes of various sizes and aspect ratios used as reference points for object detection models. These boxes are placed over the image or feature map to help the model predict the location and dimensions of objects. Rather than directly predicting bounding boxes, the model predicts offsets (shifts) relative to these anchor boxes, allowing it to adjust the box's position and size to fit the object. The model also predicts a confidence score indicating the likelihood that an object is present \cite{thinkautonomous:anchorboxes}.

\newpage

\begin{figure}[h!]
    \centering
    \includegraphics[width=0.7\linewidth]{figures/theory/image-recognition/anchor-boxes.png}
    \caption[Example of anchor boxes in object detection]{Illustration of multiple anchor boxes of different shapes and sizes centered on each grid cell in the image. These serve as initial reference boxes from which the model adjusts its predictions to fit actual objects
    \cite{thinkautonomous:anchorboxes}}
    \label{fig:anchor-box}
\end{figure}

Once multiple predictions have been made, some of them may overlap significantly. To handle this, a post-processing technique called \textbf{\gls{nms}} is used. It removes redundant or overlapping bounding boxes and keeps only the most confident prediction. The NMS algorithm works by first selecting the box with the highest confidence score. It then removes all other boxes that have a high overlap with the selected box. This process is repeated iteratively until no boxes with significant overlap remain. By applying \gls{nms}, object detectors produce cleaner and more accurate results, preventing multiple detections of the same object
\cite{thepythoncode:nms}.

\newpage

\begin{figure}[h!]
    \centering
    \includegraphics[width=0.75\linewidth]{figures/theory/image-recognition/nms.png}
    \caption[Non-maximum suppression (NMS) before and after applying the algorithm]{Illustration of NMS before and after its application. NMS retains the bounding box with the highest confidence score while suppressing those that significantly overlap with it \cite{thepythoncode:nms}}
    \label{fig:nms}
\end{figure}

\subsubsection*{Normalization}

Normalization in image processing involves scaling pixel values to a consistent range, typically [0, 1] or [-1, 1], to improve model performance. By adjusting pixel intensities from their original range (0-255) to a smaller scale, normalization ensures that the model processes inputs in a more stable and consistent manner. This accelerates model convergence. Normalization ensures that the input features are on a similar scale, aiding in more efficient and accurate learning.

\subsubsection*{Scaling}

Scaling in image processing and machine learning refers to transforming input data, such as images, to meet the specific requirements of a model. This typically involves resizing the image to a consistent size, normalizing pixel values to a standard range, and converting the image into the correct format expected by the model.





\begin{figure}[h!]
    \centering
    \includegraphics[width=0.75\linewidth]{figures/theory/image-recognition/scaling.png}
    \caption[Scaling process in image preprocessing]{Illustration of the scaling process in image preprocessing, resizing, normalizing, and converting the image format to prepare it for model input. \cite{thepythoncode:nms}}
    \label{fig:nms}
\end{figure}

Scaling process in image preprocessing]{Illustration of the scaling process in image preprocessing, including resizing, normalizing, and converting the image format to prepare it for model input.

\subsubsection*{Perspective Transformation}

When an image is taken from a tilted viewpoint, objects that are normally rectangular, such as a chessboard, appear distorted and no longer have right angles. A perspective transformation is a specific type of image warping that corrects distortions caused by viewing a flat object from an angle A perspective transformation uses mathematical techniques to map points from the distorted image back to their correct, undistorted positions \cite{nvidia:perspective-transform}.


\begin{figure}[h!]
    \centering
    \includegraphics[width=0.75\linewidth]{figures/theory/image-recognition/perspective-transformation.png}
    \caption[Perspective transformation before and after]{Perspective transformation before and after applying the algorithm, demonstrating how distortion caused by an angled viewpoint is corrected to restore the object’s original shape.}
    \label{fig:perspective-transformation}
\end{figure}





The \textbf{Euclidean distance} is the straight-line distance between two points in a plane. For two points with coordinates $\mathbf{x} = (x_1, x_2)$ and $\mathbf{y} = (y_1, y_2)$, the Euclidean distance is given by:

\[
\sqrt{(x_2 - x_1)^2 + (y_2 - y_1)^2}
\]

As shown in Figure~\ref{fig:euclidean-distance}, the Euclidean distance represents the length of the straight line connecting the two points.

\begin{figure}[h!]
    \centering
    \fbox{\includegraphics[width=0.4\linewidth]{figures/theory/math/euclidean-distance.png}}
    \caption{Illustration of Euclidean distance: the straight-line distance between two points in 2D space~\cite{geeksforgeeks:euclidean-distance}.}
    \label{fig:euclidean-distance}
\end{figure}




\subsection{Web}
\label{subsec:web}

\subsubsection*{Cross-Platform}
\label{subsubsec:corss-platform}

Cross-platform applications are software programs designed to function consistently across multiple operating systems or platforms. This includes desktop environments such as Windows, macOS, and Linux, as well as mobile platforms like iOS and Android \cite{sevenpeaks:cross-platform}.

\subsubsection*{Client-Server Architecture}
\label{subsubsec:client-server}

Client-server architecture is a network model in which multiple clients request and receive services from a centralized server over a local network or the Internet. Clients interact with the system through an application interface, while the server handles data processing. This architecture enables centralized control, scalability, and efficient resource management. See Figure~\ref{fig:client-server-architecture} \cite{liquidweb:client-server}.


\begin{figure}[h!]
    \centering
    \includegraphics[width=0.75\linewidth]{figures/theory/client-server-architecture.png}
    \caption[Client-Server Architecture]{Client-Server Architecture \cite{liquidweb:client-server}}
    \label{fig:client-server-architecture}
\end{figure}

\subsubsection*{WebSocket}
\label{subsubsec:websocket}

WebSocket is a standardized communication protocol that enables full-duplex communication over a single \gls{tcp} connection, making it well-suited for real-time web applications. Unlike traditional \gls{http} requests — which follow a request-response model — WebSocket establishes a persistent connection that allows both the client and server to send and receive data at any time. This reduces the need for polling or long polling, significantly lowering network traffic and latency. As a result, WebSocket improves the efficiency and responsiveness of data transmission, particularly in applications such as live data feeds and online games. See Figure \ref{fig:websocket-vs-http} for a comparison \cite{nodejs:websocket, apidog:websocket}.


\begin{figure}[h!]
    \centering
    \includegraphics[width=0.75\linewidth]{figures/theory/websocket-vs-http.png}
    \caption[WebSocket Connection VS HTTP Connection]{WebSocket Connection VS HTTP Connection \cite{apidog:websocket}}
    \label{fig:websocket-vs-http}
\end{figure}

\section{Design}
\label{sec:design}

\subsection{Wireframe}
\label{subsec:wireframe}

A wireframe is a rough schematic created in the early stages of digital product design to help visualize and communicate the structure of a product or website \cite{balsamiq:wireframe}. \\

The same screen can be built in a lot of different ways, but only a few of them will get your message across correctly and result in an easy-to-use software or website \cite{balsamiq:wireframe}. \\

The purpose of a wireframe is to define a skeletal layout that is easy to understand, and encourages iteration and feedback. Getting to agreement on a good interface structure is a critical part of the software design process \cite{balsamiq:wireframe}. \\

Wireframes are important because doing this work now, before any code is written and before the visual design is finalized, will save you lots of time and painful vadjustment work later \cite{balsamiq:wireframe}.

\subsection{WCAG}
\label{subsec:wcag}

\gls{wcag} provide technical specifications to improve the accessibility of websites and many other digital experiences. \cite{levelaccess:wcag}

\section{Code Quality}
\label{sec:code-quality}

\subsection{Code Review}
\label{subsec:code-review}

Code review is a critical step in the software development process, where a developer's implementation is examined by one or more peers before it is merged into an upstream branch, such as a feature branch or the main branch. This process provides a second opinion on the solution, helping to identify bugs, logic errors, uncovered edge cases, and other potential issues that may have been overlooked during development. \cite{gitlab:code-review} \\

A well-defined code review process is essential for maintaining high code quality and preventing unstable or faulty code from reaching production. By incorporating code reviews into the team's workflow, software development teams can foster continuous improvement, ensure that all code is reviewed by multiple perspectives, and reduce the risk of introducing defects into the codebase \cite{gitlab:code-review}. \\

Pull requests are a widely used mechanism to facilitate code reviews in modern version control systems. A pull request is a formal proposal to merge a set of changes from one branch into another. It provides a platform for collaborators to review, discuss, and approve changes before they are integrated into the main codebase. Pull requests also display the differences between the source and target branches, making it easier for reviewers to understand the proposed changes and provide constructive feedback \cite{github:pr}.

\subsection{Cohesion and Coupling}
\label{subsec:cohesion-and-coupling}

Cohesion and coupling are two fundamental concepts in software design that significantly impact the quality and maintainability of a system. \\

\begin{quote}
\textit{"\textbf{Cohesion} refers to the degree to which elements within a module work together to fulfill a single, well-defined purpose. \textbf{High cohesion} means that elements are closely related and focused on a single purpose, while \textbf{low cohesion} means that elements are loosely related and serve multiple purposes."} \cite{geeksforgeeks:c&c} \\
\end{quote}

\begin{quote}
\textit{"\textbf{Coupling} refers to the degree of interdependence between software modules. \textbf{Tight coupling} means that modules are closely connected and changes in one module may affect other modules. \textbf{Loose coupling} means that modules are independent, and changes in one module have little impact on other modules."} \cite{geeksforgeeks:c&c} \\
\end{quote}

\begin{figure}[h!]
    \centering
    \subfloat{{\includegraphics[width=0.5\linewidth]{figures/theory/cohesion.png}}}
    \subfloat{{\includegraphics[width=0.5\linewidth]{figures/theory/coupling.png}}}

    \caption[Cohesion \& Coupling]{Cohesion \& Coupling \cite{geeksforgeeks:c&c}}
    \label{fig:cohesion-coupling}
\end{figure}

Both cohesion and coupling are important factors in determining the maintainability, scalability, and reliability of a software system. Tight coupling and low cohesion often result in systems that are difficult to change, test, and debug. Conversely, loose coupling and high cohesion contribute to systems that are modular, flexible, and easier to improve over time. An illustration of this concept is shown in Figure \ref{fig:cohesion-coupling} \cite{geeksforgeeks:c&c}.

\newpage

\subsection{Documentation}
\label{subsec:documentation}

Documentation is a critical component of software development, consisting of written materials that accompany a software program. It serves as a comprehensive reference for all stakeholders involved in the project, including developers, testers, and end-users. Documentation can take various forms, such as \gls{api} documentation, build instructions, user manuals, or internal design specifications. Its purpose is to provide clarity, facilitate understanding, and ensure the effective use and maintenance of the software \cite{geeksforgeeks:doc}. \\

High-quality documentation is essential for the success of any software project. It simplifies onboarding for new team members, aids in troubleshooting and debugging, and ensures that the software can be maintained and extended over time. Moreover, well-documented code and systems reduce the risk of knowledge loss when team members change or when revisiting older parts of the codebase \cite{geeksforgeeks:doc}. \\

In modern software development, documentation is not merely an afterthought but an integral part of the development process. It should be created and maintained alongside the code, ensuring that it remains accurate, up-to-date, and accessible to all relevant parties.

\subsection{Testing}
\label{subsec:testing}

\subsubsection*{Virtual Machine}
\label{subsubsec:virtual-machine}

Virtualization refers to the creation of a software-based — or "virtual" — version of a computer, with dedicated allocations of \gls{cpu}, memory, and storage drawn from a physical host machine or a remote server, such as one in a cloud provider's data center. A \gls{vm} is essentially a computer file that behaves like an independent computer system. It operates within a window as a separate computing environment, isolated from the host operating system. This isolation ensures that the software running inside the \gls{vm} cannot interfere with the host machine’s primary system \cite{microsoft:virtual-machine}.

\subsubsection*{Unit Testing}
\label{subsubsec:unit-testing}

Unit testing is a fundamental software testing technique in which individual units or components of a software application are tested in isolation. A unit typically refers to the smallest testable part of a program, such as a function, method, or class. The goal of unit testing is to validate that each unit performs as expected under various conditions, ensuring its correctness and reliability \cite{geeksforgeeks:unit-test}. \\

By identifying and addressing bugs early in the development cycle, unit testing significantly enhances code quality and reduces the cost of fixing issues later in the process. It is a core practice in \gls{tdd}, where tests are written before the actual code, promoting a disciplined approach to development and ensuring that the code meets its requirements from the outset \cite{geeksforgeeks:unit-test}. \\

Unit testing also contributes to the maintainability and scalability of a software system. Well-tested units are easier to refactor, extend, and integrate into larger systems, as their behavior is clearly defined and verified. Additionally, unit tests serve as living documentation, providing insights into how individual components are intended to function.

\subsubsection*{Usability Testing}
\label{subsubsec:usability-testing}

Usability testing is a critical aspect of software testing that focuses on evaluating a system from the perspective of an end user. Its primary goal is to determine how easily and effectively users can interact with the system to achieve their objectives. This type of testing involves observing representative users as they navigate through the product, identifying areas of confusion, difficulty, or inefficiency in the user experience \cite{geeksforgeeks:user-test}. \\

During usability testing, both qualitative and quantitative data are collected to assess the product's functionality and user satisfaction. Qualitative data, such as user feedback and observations, provides insights into user behavior and preferences. Quantitative data, such as task completion rates and time-on-task, offers measurable metrics to evaluate performance. Based on the findings, a detailed report is generated, outlining necessary improvements to enhance the product's usability \cite{geeksforgeeks:user-test}. \\

The ultimate goal of usability testing is to identify pain points in the user experience and uncover opportunities for improvement. By understanding how users interact with the product and where they encounter challenges, developers can make informed decisions to refine the design, improve functionality, and increase overall user satisfaction \cite{geeksforgeeks:user-test}.

\subsection{Type Safety}
\label{subsec:type-safety}

Type safety is a fundamental concept in software development that ensures the correctness of a codebase by detecting type-related errors during the development process. In dynamically typed languages, such as JavaScript, variables can be assigned values of any type, which often leads to subtle and hard-to-detect bugs. These bugs can be time-consuming to diagnose and resolve, particularly in large or complex codebases. Type safety addresses this issue by enforcing strict type constraints, thereby reducing the likelihood of runtime errors and improving overall code reliability \cite{dev:type-safety}.

\subsubsection*{Key Pillars of Type Safety}
\label{subsubsec:type-safety-pillars}

\begin{itemize}
    \item \textbf{Reliability:} Type safety acts as a protective mechanism, preventing runtime errors caused by type mismatches. By ensuring that variables and functions adhere to predefined types, it enhances the stability and reliability of applications \cite{dev:type-safety}.

    \item \textbf{Collaboration:} Explicit type declarations improve code readability and make it easier for developers to understand and work with each other's code. This fosters seamless collaboration, especially in team environments where multiple developers contribute to the same codebase \cite{dev:type-safety}.

    \item \textbf{Efficient Debugging:} Type safety enables early detection of type-related discrepancies, reducing the time and effort required for debugging. By catching errors at compile time (or during development in statically typed languages), it minimizes the risk of runtime failures and simplifies the debugging process \cite{dev:type-safety}.
\end{itemize}

In summary, type safety is a critical aspect of code quality, offering significant benefits in terms of reliability, collaboration, and debugging efficiency. Its implementation, whether through statically typed languages or tools like TypeScript, plays a vital role in modern software development practices \cite{dev:type-safety}.
