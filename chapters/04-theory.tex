\chapter{Theory}

\section{Literature Review}

The game of chess originated in India during the Gupta dynasty in the 6th century. Over 1500 years later, chess has become a globally recognized game, played competitively in 172 countries worldwide \cite{artsnculture}. \\

With the advancement of technology, various digital applications and solutions have emerged to enhance the experience of playing and analyzing chess. Prominent online platforms such as \textit{Chess.com} and \textit{Lichess.org} allow players to compete across the globe, offering features such as online matchmaking, tutorials, and game analysis. 

Additionally, physical chessboards have been modernized through the integration of digital technologies, such as \gls{rfid} tags, which enable the digitization of physical chess games. \cite{quora:shah} \\

Furthermore, advancements in \gls{ai} have led to the development of applications capable of automating tasks such as piece recognition and game digitization. One notable example is \textit{ChessCam}, a web- and mobile-based application designed for quick and efficient live chess digitization. ChessCam allows users to either upload pre-recorded videos of chess games or stream live games using a mobile phone or webcam. The application processes the video input, recognizes the moves, and generates \gls{pgn} files, which can be used for further analysis or archival purposes.

% \section{Project Management}
% % \subsection{Sprint}
% % \subsection{Issues}
% % \subsection{Meetings}

% \section{Code Quality Assurance}
% % \subsection{Testing}
% % \subsection{Code Review}
% % \subsection{Cohesion and Coupling}
% % \subsection{Documentation}


\section{Tools and Technologies}

\subsection{Version Control}

By utilizing \gls{git}, it would allows a team of developers to work on the same code of the project, maintain the code, as well as track all the modifications of the code in the project containers called \glspl{repository}. \cite{alphaefficiency:git}

\subsection{Computer Vision}

Computer vision is a field of \gls{ai} that enables computers and systems to interpret and analyze visual data and derive meaningful information from digital images, videos, and other visual inputs. \cite{google:vision} Like other types of \gls{ai}, computer vision seeks to perform and automate tasks that replicate human capabilities. In this case, computer vision seeks to replicate both the way humans see, and the way humans make sense of what they see. \cite{microsoft:vision}

\subsection{Application}

% \section{Design}
% \subsection{UX Design}
% \subsection{Wireframes}