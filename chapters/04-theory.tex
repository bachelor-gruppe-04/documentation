\chapter{Theory}

\section{Literature Review}

The game of chess originated in India during the Gupta dynasty in the 6th century. Over 1500 years later, chess has become a globally recognized game, played competitively in 172 countries worldwide \cite{artsnculture}. \\

With the advancement of technology, various digital applications and solutions have emerged to enhance the experience of playing and analyzing chess. Prominent online platforms such as \textit{chess.com} and \textit{lichess.org} allow players to compete across the globe, offering features such as online matchmaking, tutorials, and game analysis. 

Additionally, physical chessboards have been modernized through the integration of digital technologies, such as \gls{rfid} tags, which enable the digitization of physical chess games. \cite{quora:shah} \\

Furthermore, advancements in \gls{ai} have led to the development of applications capable of automating tasks such as piece recognition and game digitization. One notable example is \textit{ChessCam}, a web- and mobile-based application designed for quick and efficient live chess digitization. ChessCam allows users to either upload pre-recorded videos of chess games or stream live games using a mobile phone or webcam. The application processes the video input, recognizes the moves, and generates \gls{pgn} files, which can be used for further analysis or archival purposes.

% \section{Project Management}
% % \subsection{Sprint}
% % \subsection{Issues}
% % \subsection{Meetings}

% \section{Code Quality Assurance}
% % \subsection{Testing}
% % \subsection{Code Review}
% % \subsection{Cohesion and Coupling}
% % \subsection{Documentation}


\section{Tools and Technologies}

\subsection{Version Control}

The use of \gls{git} as a version control system enables collaborative development by allowing multiple developers to work on the same codebase simultaneously. \gls{git} facilitates the maintenance of the code and provides a comprehensive history of all modifications made to the project. These changes are tracked within project containers known as \glspl{repository}, ensuring transparency and accountability throughout the development process. \cite{alphaefficiency:git}

\subsection{Computer Vision}

Computer vision, a subfield of \gls{ai}, focuses on enabling machines to interpret, analyze, and extract meaningful information from visual data such as digital images, videos, and other visual inputs. \cite{google:vision} Similar to other domains within \gls{ai}, computer vision aims to automate tasks that traditionally require human intelligence. Specifically, it seeks to replicate the human ability to perceive visual information and derive understanding from it. \cite{microsoft:vision}

\subsection{Machine Learning}

\gls{cnn}

\subsection{Desktop Application}

% \section{Design}
% \subsection{Wireframes}
% \subsection{UX Design}

\section{Code Quality}

\subsection{Code Review}

A pull request is a proposal to merge a set of changes from one branch into another. In a pull request, collaborators can review and discuss the proposed set of changes before they integrate the changes into the main codebase. Pull requests display the differences between the content in the source branch and the content in the target branch. \\

(\url{https://docs.github.com/en/pull-requests/collaborating-with-pull-requests/proposing-changes-to-your-work-with-pull-requests/about-pull-requests})

\subsection{Cohesion and Coupling}

Coupling refers to the degree of interdependence between software modules. High coupling means that modules are closely connected and changes in one module may affect other modules. Low coupling means that modules are independent, and changes in one module have little impact on other modules. \\

Cohesion refers to the degree to which elements within a module work together to fulfill a single, well-defined purpose. High cohesion means that elements are closely related and focused on a single purpose, while low cohesion means that elements are loosely related and serve multiple purposes. \\

Both coupling and cohesion are important factors in determining the maintainability, scalability, and reliability of a software system. High coupling and low cohesion can make a system difficult to change and test, while low coupling and high cohesion make a system easier to maintain and improve. \\

(\url{https://www.geeksforgeeks.org/software-engineering-coupling-and-cohesion/#what-is-coupling-and-cohesion})

\subsection{Documentation}

\subsection{Testing}

\subsection{Type Safety}