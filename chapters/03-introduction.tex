\chapter{Introduction}

\section{Background}

Established in 1913, \aas has maintained a longstanding passion for the noble game of chess \cite{schaklag:63}. Over the years, the organization has sought innovative ways to enhance the experience of its members while preserving the integrity and tradition of the game. One of their key objectives is to digitize the process of playing and documenting chess games. Specifically, they aim to develop a cost-effective and user-friendly solution that allows players to engage in continuous gameplay on a traditional chessboard while automatically digitizing the moves into \gls{pgn} files.

\section{Project Goal}

This digitization process would enable efficient documentation of games and facilitate advanced analysis using tools such as \Gls{stockfish}. By automating the conversion of physical gameplay into digital formats, the organization hopes to eliminate the need for expensive digital chessboards, which often require complex setup procedures. Instead, the proposed solution leverages computer vision and image recognition technologies to achieve the same functionality using locally available hardware, such as a standard webcam and a personal computer.

\section{Product Requirements}



\section{Structure of the Thesis}

% \subsubsection*{Chapter 1 -- Introduction}

% Explains the background, problem, and product requirements of the project.

% \subsubsection*{Chapter 2 -- Theory}

% Encompasses the theory and concepts behind the technologies used.

% \subsubsection*{Chapter 3 -- Method}

% Contains the details about the methods used to create the solution.

% \subsubsection*{Chapter 4 -- Results}

% Presents the results that were achived.

% \subsubsection*{Chapter 5 -- Discussion}

% Reflects on the results and methods, in relation to the problem and theory.

% \subsubsection*{Chapter 6 -- Conclusion}

% Concludes the project and report.



\textbf{Chapter 1 -- Introduction} explains the background, problem, and product requirements of the project.\\

\noindent\textbf{Chapter 2 -- Theory} encompasses the theory and concepts behind the technologies used.\\

\noindent\textbf{Chapter 3 -- Method} contains the details about the methods used to create the solution.\\

\noindent\textbf{Chapter 4 -- Results} presents the results that were achived.\\

\noindent\textbf{Chapter 5 -- Discussion} reflects on the results and methods, in relation to the problem and theory.\\

\noindent\textbf{Chapter 6 -- Conclusion} concludes the project, as well as the thesis/report.



% \begin{itemize}
%     \item \textbf{Chapter 1 -- Introduction} explains the background, problem, and product requirements of the project.

%     \item \textbf{Chapter 2 -- Theory} encompasses the theory and concepts behind the technologies used.

%     \item \textbf{Chapter 3 -- Method} contains the details about the methods used to create the solution.

%     \item \textbf{Chapter 4 -- Results} presents the results that were achived.

%     \item \textbf{Chapter 5 -- Discussion} reflects on the results and methods, in relation to the problem and theory.

%     \item \textbf{Chapter 6 -- Conclusion} concludes the project, as well as the thesis/report.
% \end{itemize}



\begin{code}
\begin{lstlisting}[language=python]
def hello_world():
"""
Some form of python-documentation.
"""
    print("Hello World!")
\end{lstlisting}
\caption{A Python function}
\label{code:python}
\end{code}



\begin{code}
\begin{lstlisting}[language=java]
/**
 * Some form of Javadoc.
 */
public class Main {
    public static void main(String[] args) {
        System.out.println("Hello World!");
    }
}
\end{lstlisting}
\caption{A Java program}
\label{code:java}
\end{code}