\chapter{Introduction}

\begin{center} 
\textit{This chapter presents the background of the project, outlines the objective, defines the requirements, and lists the equipment used.}
\end{center}


\section{Background}

Chess, one of the oldest and most revered strategy games, traces its origins to the Gupta dynasty in 6th-century India. Over the past 1,500 years, it has evolved into a global phenomenon, with competitive play spanning 172 countries worldwide \cite{artsnculture}. \\

Documenting and analyzing chess games played on traditional boards often requires manual effort or expensive digital chessboards, which come with cumbersome setup procedures. Aalesunds Schaklag envisions a cost-effective and user-friendly solution to automate the digitization of chess games.  The system will enable players to enjoy uninterrupted gameplay \gls{otb} while capturing and processing their moves in real-time. This will simplify game documentation and provide players with analysis capabilities, such as integration with tools such as \gls{stockfish}.

\section{Objective}

The objective of this project is to develop a desktop application that digitizes chess games played on a traditional board. It will use image recognition technologies and a webcam to capture and interpret gameplay in real time. To achieve this, the application will generate \gls{pgn} files, which are essential for archiving and analyzing games. The application will also provide real-time visualization through a digital chessboard that dynamically updates to reflect detected moves. This solution aims to replicate the functionality of expensive digital chessboards in a more affordable and accessible manner.


\section{Requirements}

The solution will be implemented as a local application compatible with machines running Windows or Ubuntu operating systems. The application will utilize a webcam to detect the chessboard, pieces, and moves in real-time. All processing, including image recognition and move validation, will occur locally on the user's machine.

\section{Equipment}
The following equipment was used throughout the project:

\begin{itemize}
    \item Two Logitech HD Pro Webcam C920 cameras
    \item One wooden chessboard
    \item One plastic chessboard
    \item One USB hub
    \item Staunton-style chess piece sets
\end{itemize}


\section{Thesis Structure}

\begin{itemize}
    
    \item \textbf{Chapter 2 -- Theory:} Explains the technologies and methodologies used.
    
    \item \textbf{Chapter 3 -- Methods:} Describes the development process and approach.
    
    \item \textbf{Chapter 4 -- Results:} Presents the project outcomes and performance.
    
    \item \textbf{Chapter 5 -- Discussion:} Analyzes results, limitations, and future improvements.
    
    \item \textbf{Chapter 6 -- Conclusion:} Summarizes the project.

%FILL IN - Arne Styve sa vi kan gjerne legge på litt mer spesifikke detaljer om hva de ulike delene inneholder så det ikke blir generelt

    
\end{itemize}