\chapter{Introduction}

\begin{center} 
\textit{This chapter presents the background of the project, outlines the objective, defines the requirements, and lists the equipment used.}
\end{center}

\section{Background}

Chess, one of the oldest and most revered strategy games, traces its origins to the Gupta dynasty in 6th-century India. Over the past 1,500 years, it has evolved into a global phenomenon, with competitive play spanning 172 countries worldwide \cite{artsnculture}. \\

Documenting and analyzing chess games played on traditional boards often requires manual effort or expensive digital chessboards, which come with cumbersome setup procedures. Aalesunds Schaklag envisions a cost-effective and user-friendly solution to automate the digitization of chess games.  The system will enable players to enjoy uninterrupted gameplay \gls{otb} while capturing and processing their moves in real time. This will simplify game documentation and provide players with analysis capabilities, such as integration with tools such as \gls{stockfish}.

\newpage

\section{Objective}

The objective of this project is to develop a desktop application that digitizes chess games played on a traditional board. It will use image recognition technologies and a camera to capture and interpret gameplay in real time. This solution aims to replicate the functionality of expensive digital chessboards in a more affordable and accessible manner.

\begin{itemize}
    \item \textbf{Automatic Digitization:} The application will generate \gls{pgn} files from the captured gameplay, enabling efficient documentation and archival of chess games.

    \item \textbf{Real-Time Visualization:} A digital chessboard will be displayed within the application, updating dynamically to reflect the moves detected by the camera.

    \item \textbf{Local Execution:} All processing occurs directly on the user’s device, ensuring privacy and eliminating reliance on external services.
\end{itemize}

\section{Motivation}

The team chose to work on the project \textbf{Live Digitization of Chess Games}, as it combines our shared interests in both chess and technology. All group members have experience with chess, whether as a hobby or through active involvement in chess communities. \\

We are particularly interested in exploring image recognition and machine learning technologies. Several of us have taken courses in machine learning and data processing, providing us with a foundation in the techniques required to analyze and interpret visual data. Developing a system capable of digitizing moves from a physical chessboard presents an exciting technical challenge and an excellent opportunity to apply our academic knowledge in a practical and meaningful way.

\section{Requirements}

The solution will be implemented as a local application compatible with machines running Windows or Ubuntu operating systems. A camera will be used to detect the physical chessboard and moves. All processing, including image recognition and move validation, will occur locally on the user's machine.

\newpage

\section{Equipment}
The following equipment was used throughout the project:

\begin{itemize}
    \item Two Logitech HD Pro Webcam C920
    \item One wooden chessboard
    \item One plastic chessboard
    \item One USB-hub
    \item Staunton-style chess piece sets
\end{itemize}

\section{Thesis Structure}

\begin{itemize}
    
    \item \textbf{Chapter 2 -- Theory:} Presents the theoretical background, including \gls{ai}, object detection, image processing, web architecture, design principles, and key software engineering concepts.
    
    \item \textbf{Chapter 3 -- Methods:} Describes the technical implementation and development process. It details how the machine learning models were used and combined, the design of the user interfaces, the tools and platforms chosen, the technology stack, and the testing methodology.
    
    \item \textbf{Chapter 4 -- Results:} Presents the results of the project. This includes the final implementation of the system, performance metrics from model testing, an overview of the backend \gls{api} and frontend interface, and other technical achievements.
    
    \item \textbf{Chapter 5 -- Discussion:} Analyzes the strengths and limitations of the delivered product. It discusses component-specific challenges, justifies technical decisions, reflects on project management, and proposes ideas for future development.
    
    \item \textbf{Chapter 6 -- Conclusion:} Summarizes the outcomes of the project and the extent to which the objectives and requirements were met.

    \item \textbf{Chapter 7 -- Societal Impact:} Reflects on the ethical, economic, and environmental considerations of the solution, including its sustainability and broader implications for chess digitization.
    
\end{itemize}