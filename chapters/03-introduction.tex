\chapter{Introduction}

\begin{center} \textit{This chapter introduces the background and motivation for the project, outlines the primary objectives, and defines the requirements. It establishes the context for the research and development efforts undertaken in this thesis.}
\end{center}

\section{Background}

Chess, one of the oldest and most revered strategy games, traces its origins to the Gupta dynasty in 6th-century India. Over the past 1,500 years, it has evolved into a global phenomenon, with competitive play spanning 172 countries worldwide \cite{artsnculture}. \

Documenting and analyzing chess games played on traditional boards often requires manual effort or expensive digital chessboards, which come with cumbersome setup procedures. Aalesunds Schaklag envisions a cost-effective and user-friendly solution to automate the digitization of chess games.  The system will enable players to enjoy uninterrupted gameplay on a traditional chessboard while capturing and processing their moves in real-time. This will simplify game documentation and provide players with advanced analysis capabilities, such as integration with tools like \gls{stockfish}.

\section{Objective}

The primary objective of this project is to develop a desktop application capable of digitizing chess games played on a traditional board. Using image recognition technologies, the application will capture and interpret gameplay in real-time via a standard webcam connected to the user's machine. It will generate \gls{pgn} files for efficient documentation and archival of chess games, while also providing real-time visualization through a digital chessboard that dynamically updates to reflect detected moves. This solution aims to replicate the functionality of expensive digital chessboards in a more cost-effective and accessible manner.


\section{Requirements}

The solution will be implemented as a local application compatible with machines running Windows or Ubuntu operating systems. The application will utilize a standard webcam to detect the chessboard, pieces, and moves in real-time. All processing, including image recognition and move validation, will occur locally on the user's machine. Additionally, the application will generate \gls{pgn} files for documenting games and \gls{fen} strings for integration with the \gls{stockfish} \gls{api} for move analysis.

\section{Equipment}
The following equipment was used throughout the project:

\begin{itemize}
    \item Two Logitech HD Pro Webcam C920 cameras
    \item One wooden chessboard
    \item One plastic chessboard
    \item One USB hub
    \item Staunton-style chess piece sets
\end{itemize}


\section{Thesis Structure}

\begin{itemize}
    
    \item \textbf{Chapter 2 -- Theory:} Explains the technologies and methodologies used.
    
    \item \textbf{Chapter 3 -- Methods:} Describes the development process and approach.
    
    \item \textbf{Chapter 4 -- Results:} Presents the project outcomes and performance.
    
    \item \textbf{Chapter 5 -- Discussion:} Analyzes results, limitations, and future improvements.
    
    \item \textbf{Chapter 6 -- Conclusion:} Summarizes the project.
\end{itemize}