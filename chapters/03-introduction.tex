\chapter{Introduction}

\begin{center}
    \textit{This chapter introduces the background and motivation for the project, outlines the primary objectives, and defines the product requirements. It establishes the context for the research and development efforts undertaken in this thesis.}   
\end{center}

\section{Background}

% The game of chess originated in India during the Gupta dynasty in the 6th century. Over 1500 years later, chess has become a globally recognized game, played competitively in 172 countries worldwide \cite{artsnculture}. Established in 1913, \aas has maintained a longstanding passion for the noble game of chess \cite{schaklag:63}. Over the years, the organization has sought innovative ways to enhance the experience of its members while preserving the integrity and tradition of the game. One of their key objectives is to digitize the process of playing and documenting chess games. Specifically, they aim to develop a cost-effective and user-friendly solution that allows players to engage in continuous gameplay on a traditional chessboard while automatically digitizing the moves into \gls{pgn} files. \cite{ntnuopen:21} This digitization process would enable efficient documentation of games and facilitate advanced analysis using tools such as \gls{stockfish}. By automating the conversion of physical gameplay into digital formats, the organization hopes to eliminate the need for expensive digital chessboards, which often require complex setup procedures.

Chess, one of the oldest and most revered strategy games, traces its origins to the Gupta dynasty in 6th-century India. Over the past 1,500 years, it has evolved into a global phenomenon, with competitive play spanning 172 countries worldwide \cite{artsnculture}. Its enduring appeal lies in its unique blend of strategy, creativity, and intellectual challenge, making it a cornerstone of both recreational and competitive gaming. \\

% In 1913, Aalesunds Schaklag was founded with a deep commitment to promoting and preserving the rich traditions of chess. For over a century, the organization has championed the game, fostering a vibrant community of enthusiasts and players. As technology has advanced, Aalesunds Schaklag has sought innovative ways to modernize the chess experience while staying true to its roots. One of their key initiatives is the digitization of chess gameplay—a process that bridges the gap between traditional and digital chess. \\

Currently, documenting and analyzing chess games played on traditional boards requires manual effort or expensive digital chessboards, which often come with cumbersome setup procedures. To address this, Aalesunds Schaklag aims to develop a cost-effective and user-friendly solution that automates the digitization of chess games. By leveraging computer vision and image recognition technologies, the proposed system will enable players to enjoy uninterrupted gameplay on a traditional chessboard while automatically converting their moves into \gls{pgn} files \cite{ntnuopen:21}. This digitization process not only simplifies game documentation but also unlocks advanced analysis capabilities through tools like \gls{stockfish}. Ultimately, this innovation seeks to democratize access to digital chess tools, eliminating the need for costly equipment and making the game more accessible to players of all levels.

\section{Project Goal}

The primary objective of this project is to develop a desktop application capable of automating the digitization of chess games played on a traditional board. The application will leverage computer vision and image recognition technologies to capture and interpret gameplay in real-time using a standard webcam connected to a local machine. Key functionalities of the application include:

\begin{itemize}
    \item \textbf{Automatic Digitization:} The application will generate \gls{pgn} files from the captured gameplay, enabling efficient documentation and archival of chess games.

    \item \textbf{Real-Time Visualization:} A digital chessboard will be displayed within the application, updating dynamically to reflect the moves detected by the webcam.

    \item \textbf{Move Analysis:} By generating \gls{fen} strings, the application will integrate with the \gls{stockfish} \gls{api} to provide analysis of gameplay, including suggestions for optimal moves.
\end{itemize}

This application aims to replicate the functionality of expensive digital chessboards while offering a more cost-effective and accessible alternative. By processing all data locally, the application ensures privacy and eliminates the need for cloud-based infrastructure, making it suitable for use in various settings.

\section{Product Requirements}

The product requirements, as defined by the product owner, outline the essential features and constraints for the proposed solution:

\begin{itemize}
    \item \textbf{Application:} The solution must be implemented as a local application, ensuring compatibility with local machines running either Windows or Ubuntu operating systems.

    \item \textbf{Webcam-Based Detection:} The application must utilize a standard webcam to detect the chessboard, pieces, and moves in real-time.

    \item \textbf{Local Processing:} All data processing, including image recognition and move validation, must be performed locally on the user's machine. Cloud-based solutions are explicitly excluded.

    \item \textbf{Output Formats:} The application must generate \gls{pgn} files for game documentation and \gls{fen} strings to enable integration with the \gls{stockfish} \gls{api} for move analysis.
\end{itemize}

\section{Thesis Structure}

\begin{itemize}
    
    \item \textbf{Chapter 2 -- Theory:} Explains the technologies and methodologies used.
    
    \item \textbf{Chapter 3 -- Methods:} Describes the development process and approach.
    
    \item \textbf{Chapter 4 -- Results:} Presents the project outcomes and performance.
    
    \item \textbf{Chapter 5 -- Discussion:} Analyzes results, limitations, and future improvements.
    
    \item \textbf{Chapter 6 -- Conclusion:} Summarizes the project.
\end{itemize}





%%%%% Examples on Code and Figure

% \begin{code}
%     \centering
%     \lstinputlisting[language=python,
%     firstline=1,
%     lastline=5]{code/example.py}
%     \caption{A \gls{python} code}
%     \label{code:python}
% \end{code}

% \begin{figure}
%     \centering
%     \includegraphics[width=0.5\linewidth]{figures/ntnu_basic.png}
%     \caption{Some caption}
%     \label{fig:enter-label}
% \end{figure}