\chapter{Introduction}

\section{Background}

Established in 1913, \aas has maintained a longstanding passion for the noble game of chess \cite{schaklag:63}. Over the years, the organization has sought innovative ways to enhance the experience of its members while preserving the integrity and tradition of the game. One of their key objectives is to digitize the process of playing and documenting chess games. Specifically, they aim to develop a cost-effective and user-friendly solution that allows players to engage in continuous gameplay on a traditional chessboard while automatically digitizing the moves into \gls{pgn} files. \cite{ntnuopen:21} This digitization process would enable efficient documentation of games and facilitate advanced analysis using tools such as \gls{stockfish}. By automating the conversion of physical gameplay into digital formats, the organization hopes to eliminate the need for expensive digital chessboards, which often require complex setup procedures.

\section{Project Goal}

The primary objective of this project is to develop a desktop application capable of automating the digitization of chess games played on a traditional board. The application will leverage computer vision and image recognition technologies to capture and interpret gameplay in real-time using a standard webcam connected to a local machine. Key functionalities of the application include:

\begin{itemize}
    \item \textbf{Automatic Digitization:} The application will generate \gls{pgn} files from the captured gameplay, enabling efficient documentation and archival of chess games.

    \item \textbf{Real-Time Visualization:} A digital chessboard will be displayed within the application, updating dynamically to reflect the moves detected by the webcam.

    \item \textbf{Move Analysis:} By generating \gls{fen} strings, the application will integrate with the \gls{stockfish} \gls{api} to provide analysis of gameplay, including suggestions for optimal moves.
\end{itemize}

% This application aims to replicate the functionality of expensive digital chessboards while offering a more cost-effective and accessible alternative. By processing all data locally, the application ensures privacy and eliminates the need for cloud-based infrastructure, making it suitable for use in various settings.

\section{Product Requirements}

The product requirements, as defined by the product owner, outline the essential features and constraints for the proposed solution:

\begin{itemize}
    \item \textbf{Application:} The solution must be implemented as a local application, ensuring compatibility with local machines running either Windows or Ubuntu operating systems.

    \item \textbf{Webcam-Based Detection:} The application must utilize a standard webcam to detect the chessboard, pieces, and moves in real-time.

    \item \textbf{Local Processing:} All data processing, including image recognition and move validation, must be performed locally on the user's machine. Cloud-based solutions are explicitly excluded.

    \item \textbf{Output Formats:} The application must generate \gls{pgn} files for game documentation and \gls{fen} strings to enable integration with the \gls{stockfish} \gls{api} for move analysis.
\end{itemize}

\section{Structure of the Thesis}

\begin{itemize}
    \item \textbf{Chapter 1 -- Introduction:} This chapter introduces the background and motivation for the project, outlines the primary objectives, and defines the product requirements. It establishes the context for the research and development efforts undertaken in this thesis.
    
    \item \textbf{Chapter 2 -- Theory:} This chapter delves into the theoretical foundations of the technologies and methodologies employed in the project. It covers key concepts such as computer vision, image recognition, and the integration of chess analysis tools like \gls{stockfish}.
    
    \item \textbf{Chapter 3 -- Method:} Here, the development process and methodologies are described in detail. This includes the design and implementation of the desktop application, the selection of tools and frameworks, and the approach to solving the problem.
    
    \item \textbf{Chapter 4 -- Results:} This chapter presents the outcomes of the project, including the functionality and performance of the developed application. It highlights the achieved results in relation to the project goals and requirements.
    
    \item \textbf{Chapter 5 -- Discussion:} The results are critically analyzed and discussed in this chapter. It explores the implications of the findings, identifies limitations, and suggests potential areas for future improvement.
    
    \item \textbf{Chapter 6 -- Conclusion:} The final chapter summarizes the project, revisits the objectives, and reflects on the overall success of the solution. It also provides concluding remarks and discusses the broader impact of the work.
\end{itemize}





%%%%% Examples on Code and Figure

% \begin{code}
%     \centering
%     \lstinputlisting[language=python,
%     firstline=1,
%     lastline=5]{code/example.py}
%     \caption{A \gls{python} code}
%     \label{code:python}
% \end{code}

% \begin{figure}
%     \centering
%     \includegraphics[width=0.5\linewidth]{figures/ntnu_basic.png}
%     \caption{Some caption}
%     \label{fig:enter-label}
% \end{figure}