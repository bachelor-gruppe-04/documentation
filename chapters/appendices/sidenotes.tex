\chapter{Side-note}

\subsection*{Struktur}

\begin{itemize}
    \item Sammendrag: Formål, metode, resultater, betydning - "Rapporten viser at ..."

    \itemm Innholdsfortegnelse

    \item Innledning: viser sensor om kandidatene forstår oppgaven.
    
    \item Teori: Relevant teori of forskningslitteratur m/referanse - brukes i resultat og drøfting

    \item Metode: a) Hvordan jobber dere | b) Teknisk/Vitenskapelig metope (bruker undersøkelser, kvantitativ metoder, eksprementer)

    \item Resultat: Svare på 'innledningen'. Følger av 'Metode'. Referere til 'Teori'. "Nakne" resultater

    \item Diskusjon: AV kvalitet på rsultatet; drøfte metode (a og b). Analyse. Hva betyr resultatene? \textbf{(Viktig del for å vise \textit{Selvstendighet}!)}

    \item Konklusjon: Veldig kort. Gjenta viktigste fra resultat \& drøfting. + Hva dere (som gruppe) har lært + videre arbeid? \textbf{(Minst viktigste del av hele rapporten...!)}. Tvunget til å skrive kort om arbeidet. \textbf{ALDRI NOE NYTT SKAL INTRODUSERES HER!!!}

    \item Referanser/Biblografi

    \item Vedlegg: Utregninger, dataset, tegninger, kildekoder ... sjekk med faglærer
\end{itemize}

\subsection*{Innledning}

\begin{itemize}
    \item Inn i tema (Fugleperspektiv) - Få setninger

    \item Problemstilling (Helheten) - Viktig!

    \item Oppgaven ("Vi skal teste, designe, bygge, utvikle ..." operativt - det er dette dere skal levere
\end{itemize}

Skriv en god grundig problemstilling. \\

Problemstillinger inkluderer interessante faktorer of sammenhenger, sammenhener som (kausalitet, systematikk, kronologi, forutsetninger, korrelasjoner) \\

Vise helheten oppgeven er en del av. 

Vise at dere kan analysere et problem. 

Vise at dere kan identifisere faktorer.

Vise at dee kan beherske terminologi.

Vise at dere gar oversikt over faget.

\subsection*{Teori}

\begin{itemize}
    \item Referere kort til teori of relevant litteratur - ikke gjengi i detalj.
    \item Husk referanse til bok og artikkel
    \item Ikke skriv sammendtag av lærebok
    \item \textit{Det du tranger fpr å løse oppgaven}
    \item domenesprsifikk teori først (Sjakk i dette tilfellet)
\end{itemize}

\subsubsection*{Illustrasjoner}

\begin{itemize}
    \item Alle illustrasjoner \textbf{må} ha billedtekst 
    \item \textbf{Skal} være nummerert 
    \item \textbf{Skal} fortelle leser hva bildet viser, forklare symboler osv. 
    \item \textbf{Skal} refereres til i teksten - aldri stå alene. 
    \item Husk event. copyright 
    \item Legg teksten "rundt" 
    \item "Illustrasjon" er ikke "kilde"
\end{itemize}

\subsection*{Avsnittet}

\textit{"Kan du skrive avsnitt - kan du skrive alt"} 

\textit{Teksten er en skuffeseksjon og hvert avsnitt er en skuff}

\subsubsection*{Skuffen}

\begin{itemize}
    \item Inneholder kun det som hører sammen
    \item Må ha en temasetning (merkelapp)
    \item Må være systematisk organiser, med en \textit{indre orden}
\end{itemize}
