\chapter{Results}

\begin{center}
    \textit{This chapter presents the outcomes of the project, including the functionality and performance of the developed application. It highlights the achieved results in relation to the project goals and requirements.}
\end{center}

\section{Overview of Delivered Product}
The delivered product refers to the final version of the application submitted at the conclusion of this Bachelor’s thesis. The application was developed at the request of the product owner and based on the requirements provided. The initial description served as a general concept rather than a detailed specification, which means that decisions about design, architecture, and technology were left to the development team. Throughout the development process, ideas and design choices were discussed with the product owner and approved during regular meetings. \\

The application is developed as an open source project with the intention that the code can be reused, maintained, and improved by others. All components are built using available and open source technologies, ensuring that the system can be installed on local hardware without the need for licensed software or cloud services. This aligns with the requirement that all processing be performed locally, using common hardware such as a webcam connected to a local machine running Windows or Ubuntu. \\

The project follows clear coding standards and best practices, making the code easy to understand and extend. All functionalities are well documented, enabling future developers to build upon the existing solution, for example, to support new chess variants, integrate additional hardware, or improve the recognition models. To support further development, the code is published in a public Git repository without restrictions on reuse or modification. \\

The application is a local, camera-based system for digitalizing over-the-board chess games in real time. It detects piece movements on a physical chessboard using image recognition and converts the game into a PGN file, which can be viewed or analyzed digitally. The system runs entirely on local hardware without reliance on cloud services and is designed to be low-cost, open source, and easy to maintain. A front-end interface allows users to follow games live, while a back-end handles board recognition and game logic. The application is built with extensibility in mind, allowing future support for additional features or chess variants.

