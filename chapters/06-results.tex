\chapter{Results}

\begin{center}
    \textit{This chapter presents the outcomes of the project, including the functionality and performance of the developed application. It highlights the achieved results in relation to the project goals and requirements.}
\end{center}

\section{Overview of Delivered Product}
The delivered product refers to the final version of the application submitted at the conclusion of this Bachelor’s thesis. The application was developed at the request of the product owner and based on the requirements provided. The initial description served as a general concept rather than a detailed specification, which meant that decisions regarding design, architecture, and technology were left to the development team. Throughout the development process, ideas and design choices were discussed with the product owner and approved during regular meetings. \\

The application was developed as an open-source project with the intention that the code can be reused, maintained, and extended by others. All components are built using open and freely available technologies, ensuring that the system can be installed on local hardware without the need for licensed software or cloud-based services. This aligns with the requirement that all processing should be performed locally, using common hardware such as a webcam connected to a local machine running either Windows or Ubuntu. \\

The project follows clear coding standards and development best practices, making the code easy to understand and modify. All functionality is well documented, enabling future developers to build upon the existing solution. For example, to support new chess variants, integrate additional hardware, or improve the recognition models. To support future development and accessibility, the source code is published in a public Git repository without restrictions on reuse or adaptation. \\

The application itself is a local, camera-based system for digitalizing over-the-board chess games in real time. It detects piece movements on a physical chessboard using image recognition and converts the game into a PGN file, which can be viewed or analyzed digitally. The system operates entirely on local hardware, with a front-end interface that allows users to follow games live and a back-end that manages board recognition and game logic. The application is designed to be low-cost, open-source, and easy to maintain, with extensibility in mind for future features.

