\chapter{Discussion}

\begin{center}
    \textit{The results are critically analyzed and discussed in this chapter. It explores the implications of the findings, identifies limitations, and suggests potential areas for future improvement.}
\end{center}

\section{Temprorary Title - Discussion of Results}

The model’s approach of selecting the most likely move based on a confidence threshold of 0.60 revealed an interesting flaw in the detection of castling. Specifically, castling moves were often misclassified because the system evaluated each piece move independently. In many cases, the model assigned a higher confidence score to a rook moving from its starting square (e.g., from \textit{h1} to \textit{f1}) than to the king's two-square movement required for castling. As a result, 8 out of 9 castling errors occurred when the rook's move was mistakenly chosen over the correct castling sequence. A possible solution is to temporarily delay the king’s movement, ensuring the rook is still present on its original square. This could help the model assign greater certainty to the actual castling pattern.

Another observation was the consistent failure point in specific openings. In several test sets, all 10 games of a given opening failed at the exact same move number. This pattern suggests that certain positions or board states consistently challenge the model, possibly due to recurring piece configurations or lighting conditions at those points in the game.

Furthermore, move detection accuracy was noticeably lower at the edges of the board. This can be attributed to several factors. First, the board warping process relies on accurate corner detection to map square centers. While inner squares were generally well-aligned, squares near the edge—particularly the corners—were more prone to misalignment. This distortion likely contributed to decreased detection reliability in those areas. Second, the camera angle used in the setup was relatively steep, making it harder to clearly capture the chess pieces closest to the lens. Third, pieces on the far side of the board appeared smaller in the frame, and the distance between square centers in the warped image was compressed. This made it more likely for a piece to be assigned to an adjacent square, particularly along the edges where space is visually tighter.

These findings highlight key limitations in the current setup and suggest concrete areas for improvement, such as refining the warping algorithm, enhancing corner detection accuracy, and adjusting the model’s treatment of multi-piece moves like castling.


\subsubsection{Different gamemodes}

The models job is to find the pieces on the board and then assign them to its respective square. This means as long as you define the starting postion or start fen, that allows you to play any the game from any posistion you want. That means you could start a match mid game or you could play other game modes. The model also has validation so that means it should be able to successfully




\section{Further development}
Throughout the duration of this project, several ideas for development emerged. These from both the team, the product owner, and external contributors. Due to the time constraints of the Bachelor’s thesis, many of these ideas were not implemented and were therefore categorized as "Further development". After the product owner's request, ensuring accurate piece recognition was prioritized over extended functionality.

\subsection{Independent application}
Future plans include a standalone version of the application with ceiling-mounted cameras at the chess club. These would connect to a PC running the system continuously, operable via a physical switch. Players could set up boards and activate the system without technical knowledge. The application would start automatically, read boards in real time, and shut down via the same switch.

\subsection{Participant Management}
Players are currently hardcoded. A future version could allow manual registration through a UI, with data stored in a database. Integration with FIDE’s API could enable automatic rating retrieval.

\subsection{Time control}
The current system lacks time control tracking. Simple estimation based on timestamps is unreliable. Options include integrating digital clocks (costly) or using camera-based OCR to read physical clocks (technically complex and error-prone). This remains a future consideration. \\

\section{Obstacles}
In addition to ideas for future development, there were also a few obstacles encountered during the development phase of this project. These challenges were addressed as they occurred, and solutions were implemented as part of the final application.

\subsection{Camera}
During the development and testing phase of the front-end application, an obstacle related to camera initialization was encountered. The application is designed to access an external USB camera by specifying a camera ID other than 0, since ID 0 typically corresponds to the system's default or built-in (dashboard) camera, which is not relevant to this project.\\

Two identical external webcams were used for this project, both connected via a USB hub. One of the cameras functioned correctly. When connected, the front-end application successfully initialized the camera feed, and the video stream was displayed on the webpage as expected.\\

However, the second camera, did not behave as expected. Despite being physically identical to the first camera and verified to work through the system's built-in camera application, the front-end application failed to display its video feed. Instead, it defaulted to camera ID 0, resulting in the system's dashboard camera being selected.\\

This issue was traced back to a cached or “ghost” device entry in the Windows Device Manager. A ghost device refers to a previously connected hardware device that is no longer physically attached to the system but whose configuration and driver information remain cached by the operating system. These leftover entries can lead to conflicts or incorrect device indexing when similar hardware is reconnected, especially when multiple identical devices are used. In this case, the system appeared to retain prior camera associations, which caused the application to misidentify or incorrectly assign the camera index.

\subsection{Special moves}
When a move is made, the front-end highlights the previous and current tiles of the moved piece. User testing with different color palettes resulted in the selection of a bright contrast color relative to the main design scheme to ensure good visibility. \\

The chessboard component manages the rules of chess, while the front-end is responsible only for storing tile positions and applying styling. In standard moves, highlighting the starting and ending tiles is sufficient. However, special moves such as \gls{castling}, pawn \gls{promotion}, and \gls{en-passant} capture require additional handling. \\

In \gls{castling}, both the king and a rook move simultaneously. Since the default highlight logic tracks only one piece, it does not correctly represent \gls{castling} moves. Additional logic was implemented to highlight the movements of both the king and rook during \gls{castling}, ensuring consistency and clarity for the user.



% Ting som nevnes andre steder i dokumentet som er flyttet hit fordi det funker bedre i diskusjon


%The ONNX format was chosen because it was framework-agnostic, making integration into different deployment environments easier.

%Metode







%(Metode, agile methodology)

%This approach supported shared decision-making and maintained consistency across the project.

%This allowed the team to identify challenges, recognize what worked well, and suggest improvements for future sprints, promoting continuous learning and better teamwork.

%"Status reports involved setting specific sprint goals, outlining the tasks each team member aimed to complete before the next meeting. These meetings also included a review of completed tasks, allowing the team to assess progress and discuss the outcomes of the previous sprint. If any tasks remained incomplete, the team identified potential causes and developed strategies to address them moving forward."

%"Retrospectives provided a chance to reflect on the team's collaboration, discussing both strengths and weaknesses in teamwork."

%"This approach supported shared decision-making and maintained consistency across the project."
