\chapter{Discussion}

\begin{center}
    \textit{The results are critically analyzed and discussed in this chapter. It explores the implications of the findings, identifies limitations, and suggests potential areas for future improvement.}
\end{center}

\section{Further development}
Throughout the duration of this project, several ideas for development emerged. These from both the team, the product owner, and external contributors. Due to the time constraints of the Bachelor’s thesis, many of these ideas were not implemented and were therefore categorized as "Further development". After the product owner's request, ensuring accurate piece recognition was prioritized over extended functionality.

\subsection{Independent application}
During the meeting with the product owner at Aalesund Schaklag's location, the product owner and the leader of the chess club discussed potential future developments. One idea was to implement a finished version of the application where cameras are permanently mounted on the ceiling at the club. Since the chess club typically maintains standard table arrangement for chessboards, the cameras would not need to be moved. \\

In this setup, the cameras would be connected to a computer located in a locked room, running the application continuously. The proposed solution involves connecting the system to a physical switch, enabling users to easily power the cameras and application on or off as needed. \\

A typical case would involve a group of chess players organizing a unofficial tournament. Upon arrival, the tables would already be positioned correctly under the cameras. The participants would set up their chessboards, pieces, and clocks, and then activate the system by turning on the switch. Once activated, the application would automatically start, and the cameras would begin reading the boards. Games would be tracked and displayed through the application in real time. Boards set back to their initial position would be recognized as reset. \\

After the tournament concludes, users could simply turn off the switch to shut down the system. This design enables the application to be used independently, without requiring technical expertise. It is intended to be accessible to both young adults and elderly users. Additionally, mounting the cameras on the ceiling contributes to enhanced security for the chess equipment.

\subsection{Analyze chess games}
An additional development idea involves giving both players and spectators ability to analyze completed chess games. Currently, the application generates a PGN (Portable Game Notation) file once a game is completed. This file can be imported into external platforms such as Lichess, where users can step through each move and receive insights such as the opening name, evaluation of moves, and suggested best plays. \\

A proposed enhancement is to integrate the application directly with the Stockfish chess engine. By doing so, real-time analysis could be made available during ongoing games. This would allow spectators to view the current game evaluation — for example, indicating which player has the advantage at a given position. A common and effective way to display this is through an evaluation bar, which shifts visually based on the balance of the position, providing intuitive feedback to viewers. \\

This feature could be integrated into both the tournament- and board-view, displaying evaluations for each board during live play. Such functionality would enrich the viewing experience for spectators and could also be useful for players who wish to review their games afterward. After the match, players could access a breakdown of the game with move-by-move assessments and suggestions for improvement. \\

Integrating Stockfish would also open possibilities for additional automated insights, such as identifying blunders, mistakes, and inaccuracies, or offering engine-generated commentary. This would enhance the educational value of the application and make it more engaging for a broader range of users, including students, coaches, and casual players.

\section{Obstacles}
\subsection{Camera}
During the development and testing phase of the front-end application, an obstacle related to camera initialization was encountered. The application is designed to access an external USB camera by specifying a camera ID other than 0, since ID 0 typically corresponds to the system's default or built-in (dashboard) camera, which is not relevant to this project.\\

Two identical external webcams were used for this project, both connected via a USB hub. One of the cameras functioned correctly. When connected, the front-end application successfully initialized the camera feed, and the video stream was displayed on the webpage as expected.\\

However, the second camera, did not behave as expected. Despite being physically identical to the first camera and verified to work through the system's built-in camera application, the front-end application failed to display its video feed. Instead, it defaulted to camera ID 0, resulting in the system's dashboard camera being selected.\\

This issue was traced back to a cached or “ghost” device entry in the Windows Device Manager. A ghost device refers to a previously connected hardware device that is no longer physically attached to the system but whose configuration and driver information remain cached by the operating system. These leftover entries can lead to conflicts or incorrect device indexing when similar hardware is reconnected, especially when multiple identical devices are used. In this case, the system appeared to retain prior camera associations, which caused the application to misidentify or incorrectly assign the camera index.

\subsection{Special moves}
When a move is made, the front-end highlights the previous and current tiles of the moved piece. User testing with different color palettes resulted in the selection of a bright contrast color relative to the main design scheme to ensure good visibility. \\

The chessboard component manages the rules of chess, while the front-end is responsible only for storing tile positions and applying styling. In standard moves, highlighting the starting and ending tiles is sufficient. However, special moves such as castling, pawn promotion, and en passant capture require additional handling. \\

In castling, both the king and a rook move simultaneously. Castling kingside (short castling) moves the king toward the board's edge, while castling queenside (long castling) moves the king toward the center. Since the default highlight logic tracks only one piece, it does not correctly represent castling moves. Additional logic was implemented to highlight the movements of both the king and rook during castling, ensuring consistency and clarity for the user.
