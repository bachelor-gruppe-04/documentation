\chapter{Discussion}

\begin{center}
    \textit{The results are critically analyzed and discussed in this chapter. It explores the implications of the findings, identifies limitations, and suggests potential areas for future improvement.}
\end{center}

\section{Future Work}

\section{Obstacles}
\subsection{Camera}
During the development and testing phase of the front-end application, an obstacle related to camera initialization was encountered. The application is designed to access an external USB camera by specifying a camera ID other than 0, since ID 0 typically corresponds to the system's default or built-in (dashboard) camera, which is not relevant to this project.\\

Two identical external webcams were used for this project, both connected via a USB hub. One of the cameras functioned correctly. When connected, the front-end application successfully initialized the camera feed, and the video stream was displayed on the webpage as expected.\\

However, the second camera, did not behave as expected. Despite being physically identical to the first camera and verified to work through the system's built-in camera application, the front-end application failed to display its video feed. Instead, it defaulted to camera ID 0, resulting in the system's dashboard camera being selected.\\

This issue was traced back to a cached or “ghost” device entry in the Windows Device Manager. A ghost device refers to a previously connected hardware device that is no longer physically attached to the system but whose configuration and driver information remain cached by the operating system. These leftover entries can lead to conflicts or incorrect device indexing when similar hardware is reconnected, especially when multiple identical devices are used. In this case, the system appeared to retain prior camera associations, which caused the application to misidentify or incorrectly assign the camera index.

\subsection{Detection}

\subsubsection*{Visibility}

Hovering hand over the chess board hid the chess pieces, making the camera not see the pieces.

\subsubsection*{Moves}

\begin{itemize}
    \item \Gls{castling} ...

    \item \Gls{en-passant} ...
\end{itemize}