\chapter*{Abstrakt}
\addcontentsline{toc}{chapter}{Abstrakt}

Dette prosjektet, gjennomført i samarbeid med \textbf{Aalesunds Schaklag}, hadde som mål å utvikle et automatisert system for digitalisering av sjakkpartier spilt over brett. Systemet gjør det mulig for spillere å bruke et tradisjonelt fysisk brett, samtidig som trekkene deres blir fanget opp og registrert i sanntid. Bildegjenkjenning blir brukt for å oppdage brettets tilstand og validere trekk i henhold til sjakkreglene. De digitaliserte partiene blir vist via et frontend-grensesnitt, og brukere kuan hente ut PGN (trekkhistorikk) for pågående partier. \\

Løsningen ble designet for å kjøre fullstendig på lokal maskinvare, ved bruk av et USB-tilkoblet webkamera og en maskin med Windows eller Ubuntu operativsystem, uten avhengighet til skytjenester. I motsetning til kommersielle digitale sjakkbrett, som ofte er kostbare og kompliserte å sette opp, tilbyr dette systemet et rimelig og lettdistribuerbart alternativ. \\

Denne løsningen gjør det mulig å følge sjakkpartier i sanntid, validere trekk automatisk, og eksportere PGN-filer for videre analyse. Utviklingen fulgte smidige prinsipper og benyttet verktøy som Python og LeYOLO for å levere en funksjonell og skalerbar prototype, samt React og TypeScript for frontend-utvikling. Prosjektet viser at rimelig og lokalt driftet digitalisering av sjakkpartier over brett er både gjennomførbart og praktisk for både klubber og enkeltpersoner.

\chapter*{Abstract}
\addcontentsline{toc}{chapter}{Abstract}
This project, conducted in collaboration with \textbf{Aalesunds Schaklag}, aimed to develop an automated system to digitize over-the-board chess games. The system allow players to use a traditional physical board while their moves are captured and recorded in real time. Image recognition is utilized to detect board states and validate moves according to chess rules. The digitized games are displayed through a frontend interface, allowing users to retrieve the PGN (move history) of the ongoing game \\

The solution was designed to run entirely on local hardware, using a USB-connected webcam and a Windows or Ubuntu machine operating systems, with no reliance on cloud services. Unlike commercial digital chessboards, which are often costly and complex to set up, this system provids a cost-effective and easy-to-deploy alternative. \\

The final solution enables real-time game tracking, automatic move validation, and PGN export for analysis. Development followed agile principles and employed tools such as Python and LeYOLO to deliver a functional and scalable prototype, alongside React and TypeScript for the frontend interface. The project demonstrated that affordable, locally hosted digitization of over-the-board chess games is feasible and practical for clubs and individual users.