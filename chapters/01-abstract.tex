\chapter*{Abstract}
\addcontentsline{toc}{chapter}{Abstract}

% \begin{itemize}
%     \item research question
%     \item methods
%     \item results
%     \item conclusion
% \end{itemize}

The purpose of this project is to develop an automated solution for \aas that enables continuous chess gameplay on a regular board while simultaneously digitizing the game into \gls{pgn} files. The digitized games can be made accessible via an \gls{api} or by streaming moves as events on a message queue. The system leverages image recognition to identify the board and pieces, coupled with real-time validation of moves to ensure compliance with chess rules.\\

The solution is designed to operate on local hardware, typically involving a \acrshort{usb}-connected webcam and a local machine (Windows or Ubuntu), ensuring all processing is performed locally without reliance on cloud infrastructure. While digital chessboards capable of similar functionality exist, they are often costly and require significant setup time for data transmission. This project aims to provide a cost-effective and efficient alternative by automating the digitization process.\\

The development process followed \gls{agile} methodologies, utilizing technologies such as \acrshort{leyolo} and \gls{python} to implement the system. This approach ensured iterative progress, adaptability to changing requirements, and the delivery of a functional and scalable solution.\\

\textbf{*recap of the result is written here*}\\

\textbf{*recap of the conclusion is written here*}