\chapter*{Abstrakt}
\addcontentsline{toc}{chapter}{Abstrakt}

Formålet med dette prosjektet er å utvikle en automatisert løsning for \textbf{Aalesunds Schaklag} som muliggjør kontinuerlig sjakkspilling på et vanlig brett mens du samtidig digitalisere spillet til \gls{pgn}-filer. Digitaliserte spill gjøres tilgjengelige via et \gls{api} ved å strømme bevegelser som hendelser i en meldingskø. Systemet utnytter bildegjenkjenning for å identifisere brettet og brikkene, kombinert med sanntidsvalidering av trekk for å sikre overholdelse av sjakkreglene. \\

Løsningen er designet for å operere på lokal maskinvare, vanligvis med et \acrshort{usb}-tilkoblet webkamera og en lokal maskin (Windows eller Ubuntu), som sikrer at behandling utføres lokalt uten avhengighet av skyinfrastruktur. Mens digitale sjakkbrett som er i stand til lignende funksjonalitet finnes, er de kostbare og er omfattende å sette opp. Dette prosjektet har som mål å gi et kostnadseffektivt og effektivt alternativ ved å automatisere digitaliseringsprosessen. \\

Utviklingsprosessen fulgte smidige metoder, ved å bruke teknologier som \gls{leyolo} og \gls{python} for å implementere systemet. Denne tilnærmingen sikret iterativ fremgang, tilpasningsevne til endrede krav, og levering av en funksjonell og skalerbar løsning. \\

\textbf{*oppsummering av resultat skrives her*}\\

\textbf{*oppsummering av konklusjon skrives her*}


\newpage


\chapter*{Abstract}
\addcontentsline{toc}{chapter}{Abstract}

The purpose of this project is to develop an automated solution for \textbf{Aalesunds Schaklag} that enables continuous chess gameplay on a regular board while simultaneously digitizing the game into \gls{pgn} files. The digitized games can be made accessible via an \gls{api} or by streaming moves as events on a message queue. The system leverages image recognition to identify the board and pieces, coupled with real-time validation of moves to ensure compliance with chess rules.\\

The solution is designed to operate on local hardware, typically involving a \acrshort{usb}-connected webcam and a local machine (Windows or Ubuntu), ensuring all processing is performed locally without reliance on cloud infrastructure. While digital chessboards capable of similar functionality exist, they are often costly and require significant setup time for data transmission. This project aims to provide a cost-effective and efficient alternative by automating the digitization process.\\

The development process followed \gls{agile} methodologies, utilizing technologies such as \acrshort{leyolo} and \gls{python} to implement the system. This approach ensured iterative progress, adaptability to changing requirements, and the delivery of a functional and scalable solution.\\

\textbf{*recap of the result is written here*}\\

\textbf{*recap of the conclusion is written here*}