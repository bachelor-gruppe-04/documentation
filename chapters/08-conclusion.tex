\chapter{Conclusions}

\begin{center}
    \textit{The concluding chapter summarizes the project, revisits the objectives, and reflects on the overall success of the solution. It also provides concluding remarks and discusses the broader impact of the work.}
\end{center}

The goal of this project was to develop a cost-effective and automated system for digitizing \gls{otb} chess games, in collaboration with Aalesunds Schaklag. The final product enables real-time capture, validation, and digital representation of chess games played on a physical board, with moves automatically recorded and exported in \gls{pgn} format for later analysis. \\

The system successfully utilizes computer vision and machine learning to detect piece positions, validate moves according to standard chess rules, and update a web-based frontend without requiring any user interaction during gameplay. By relying solely on local hardware and avoiding cloud services, the system offers a offline-friendly, and easily deployable alternative to expensive commercial digital boards. \\

Through the use of Python and \gls{leyolo} for image processing and React with TypeScript for the frontend, the project delivered a scalable and modular prototype that fulfills its intended purpose. The application architecture also supports future expansion, such as additional game modes, multi-board tracking, and clock integration. \\

In general, the project demonstrates that affordable and practical digitization of traditional chess games is achievable using modern tools and a well-structured development approach. The final prototype meets the needs of both chess clubs and individual users, offering a solid foundation for continued development and deployment in real-world tournament environments.