\chapter{Societal Impact}

\begin{center}
    \textit{This section discusses the environmental and economic implications of the project, including ethical considerations and sustainability. Reflects on how work can contribute to societal goals and reduce negative environmental impact.}
\end{center}

\section{Professional Ethics and Reflections}
The use of real-time video input introduces important ethical considerations related to data collection and privacy. While the system relies on live camera feeds to detect chess moves, no video data is stored or archived. All image processing takes place locally and in real time, ensuring that no identifiable information is retained beyond the moment of inference. \\

This application is collecting only the data strictly necessary for functionality, not storing sensitive data, and ensuring transparency about how data is used. Additionally, the application requires no user authentication or external communication, further reducing the risk of unauthorized data access or misuse. \\

All testing during development was carried out anonymously and was voluntary. No information about test participants was recorded or linked to personal identities, and the chess games used for evaluation were not associated with individual players. \\

\section{Economic Analysis}
Table \ref{tab:detailed-price-comparison} presents a cost comparison between a standard \gls{fide}-approved digital chess setup and the alternative solution proposed in this project. The analysis demonstrates a significant potential for cost savings without compromising the core functionality required for tournament-level monitoring. \\

The total cost of a single \gls{fide}-championship setup amounts to 17\,592.26 NOK. In contrast, the alternative configuration, which relies on a traditional wooden board, webcam-based tracking, and a lower-cost clock, totals only 4\,760.49 NOK. This represents a reduction of approximately 73\% in hardware costs per unit, amounting to a total savings of 12\,831.77 NOK. \\

These savings scale substantially with the number of boards deployed. For example, outfitting 20 boards using the standard \gls{fide} setup would cost over 350\,000 NOK, while the proposed solution would remain under 100\,000 NOK. This economic advantage not only lowers the entry barrier for smaller clubs and local organizers but also allows for more flexible budgeting in large-scale tournaments. \\

\begin{table}[h!]
\centering
\caption[Detailed cost comparison]{Detailed cost comparison between official \gls{fide}-approved equipment and this project's proposed alternative solution \cite{dgtshop:prices}.}
\label{tab:detailed-price-comparison}
\begin{tabular}{lL{4cm}rL{4cm}r}
\toprule
\textbf{Category} & \textbf{FIDE Set Item} & \textbf{Price (NOK)} & \textbf{Alternative Item} & \textbf{Price (NOK)} \\
\midrule
Board & Bluetooth e-Board Walnut & 7\,928.09 & DGT Walnut Chess Board & 1\,273.12 \\
Pieces & Official FIDE Weighted Pieces & 7\,985.96 & DGT Timeless Wooden Pieces & 1\,273.12 \\
Clock & DGT3000 FIDE Clock & 983.78 & DGT1500 Clock & 520.82 \\
Bag & Carrying Bag & 694.43 & Carrying Bag & 694.43 \\
Camera & None & 0.00 & Logitech C920 Webcam & 999.00 \\
\midrule
\textbf{Total} & & \textbf{17\,592.26} & & \textbf{4\,760.49} \\
\bottomrule
\end{tabular}
\vspace{0.3cm}

\textbf{Total Savings: 17\,592.26 - 4\,760.49 = 12\,831.77 NOK}
\end{table}

\section{Sustainability Assessment}
This project aligns with the United Nations Sustainable Development Goal 9 “Industry, Innovation and Infrastructure” and Goal 12 “Responsible Consumption and Production”. By using a standard webcam and computer vision to digitize chess games, rather than relying on expensive digital chessboards, the solution fosters innovation through accessible and low-maintenance technology (Goal 9). At the same time, it addresses responsible consumption (Goal 12) by reducing the need for specialized hardware. Instead of manufacturing and maintaining multiple digital boards, tournament organizers can rely on simple camera setups, which reduces material use, energy consumption, and electronic waste.

\section{Environmental Evaluation}
The environmental impact of the system is relatively low, due to several deliberate design choices. First, the use of existing and widely available hardware, such as standard webcams and personal computers, reduces the need for new manufacturing and transportation of specialized equipment. This limits material consumption and contributes to a lower carbon footprint compared to deploying multiple digital chessboards that require custom electronics and assembly. \\

Secondly, the system is designed to operate entirely offline and locally. All processing takes place on the user's machine, which avoids continuous network traffic, remote data storage, or the need for power-hungry cloud infrastructure. This helps reduce the overall energy footprint of the solution, especially during tournaments involving multiple boards. \\

Lastly, by not recording or storing video data, the system minimizes digital storage requirements and prevents unnecessary long-term data retention, which also indirectly reduces energy usage. These choices make the application a more sustainable alternative for chess digitization and highlight the potential of lightweight, focused software solutions in reducing environmental impact.
