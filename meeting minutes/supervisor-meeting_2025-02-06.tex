\section{Meeting with supervisor - February 06th}
\begin{tabular}{ll}
    \textbf{Date:} & Friday, February 06th 2025 \\
    \textbf{Time:} & 10.00 - 10.40\\
    \textbf{Location:} & A433 Ankeret, NTNU Ålesund \\
    \textbf{Participants:} & Birgitte Thoresen, Chris Sivert Sylte, Vegard Mytting and Saleh Abdel-Afou Alaliyat\\
\end{tabular}

\vspace{0.5cm}

\subsection{Agenda}

\begin{itemize}
    \item Selection of Language and Tools
    \item Review of Completed Work
    \item Further development and improvement
    \item Clarification of Action Item
    \item Approval of Documentation Submitted on Blackboard
    \item Subjects to be discussed with Product Owner
\end{itemize}

The Product owner has not specified which language and tools to use for this project. A majority of the team members have previous experience with the use of Python, through the course Machine Learning. After discussion with the supervisor, the different possibilities were presented. Python is the leading language used for machine learning and is additionally used in the machine learning course taught at NTNU. Therefore, the natural choice for the machine learning section of the project was to use Python. For the application part, the choice was between using Python or Java. Throughout the bachelor, development of a desktop application has been done using Java. Despite having experience with Java, the choice was Python. This, because its easier to connect the machine learning section and the application section with them both using Python. \\

Both the documentation and the application have been off to a good start. The documentation required before the start of the project is submitted on Blackboard. Such as the pre-project plan, work contract, and standard agreement. The main report is also in progress, with continuous work on the introduction, the theory and the methods, as well as the sections abstract and preface. In the application section, a simple model is created. This is an ONNX model, which recognizes the chess pieces, as well as giving a bounding box and a confidence score. Further improvement will be discussed at the meeting with the product owner on February 14. Small adjustments can be made to the model, but major changes such as retraining the model are time-consuming. The focus should be on understanding the model's fundamentals and the methods used. Ideas should be discussed, such as the ability to store or download the data. This might cause some problems, considering the privacy of the players. Another idea is to read and display the clock in real time. The focus should be on satisfying Aalesunds Schaklag's needs. They have submitted the same project twice, suggesting that the previous project did not meet their requirements. Therefore, the next meeting with the project owner should focus on clarifying their expectations and needs. \\

There are different types of action items. An action item in context of meeting notes, the action items define "what to be done?". This can include tasks for the group, the supervisor, and the product owner. For example, "The supervisor needs to review the submitted documentation before the February 15 deadline." \\

In the context of the Sprint Retrospective, action items focus on refining the work completed in the previous sprint. They should clearly define how the process can be improved, why the change is necessary, and the deadline for implementation.

\subsection{Actions}
\begin{itemize}
    \item The supervisor needs to review the documentation that has been submitted on Blackboard. This includes the pre-project plan, work contract, and standard agreement. If some adjustments are needed, the group should be noticed before the deadline at \textbf{February 15}.
    \item In the meeting with the product owner on February 14, the focus should be on clarifying their expectations and needs. This includes the selection of languages and tools, the expected functionality and features, as well as the design and interface of the application.
\end{itemize}

\subsection{Other}
The current meeting time, Fridays at 10.00, conflicts with some group members' schedules. The meeting will still be held every other Friday, but the new time will be Fridays at 12.00. The location remains the same, in the room A433 Ankeret at NTNU Aalesund. \\

If possible, organizing a local chess tournament using the solution would be beneficial for the project. This would involve collaboration between the bachelor group and NTNUI Sjakk Aalesund, providing both testing opportunities and valuable feedback. The tournament should be held in April, allowing time for any necessary adjustments.