\section{Meeting with supervisor - January 24th}
\begin{tabular}{ll}
    \textbf{Date:} & Friday, January 24th 2025 \\
    \textbf{Time:} & 10.00 - 11.00\\
    \textbf{Location:} & A433 Ankeret, NTNU Ålesund \\
    \textbf{Participants:} & Birgitte Thoresen, Chris Sivert Sylte, Vegard Mytting and Saleh Abdel-Afou Alaliyat\\
\end{tabular}

\vspace{0.5cm}

The standard agreement needs to be signed before final delivery scheduled for 24th February. Signatures from the team members and the supervisor are still pending. Product owners signature with Adobe Acrobat is not valid, and needs to be signed again. The signature requires identification, so this type if e-signature is not enough. A digital signature method could be explored such as though https://signering.posten.no/, but identification will be required for verification. If a physical signature is used, the signed document will need to be scanned and delivered digitally. Using a tablet or another digital device for signing is also an option to discuss with Arne. There should be no changes are made to the document after it has been sent to and approved by the supervisor.\\

Regarding the meeting with Arne, he has been invited to meet in his office, but the date for this meeting has not yet been decided. His office is located in the centrum of Aalesund. \\

Planned documentation has been mostly completed, and research into implementation methods has been conducted. Equipment, including a camera, chessboard, and chess pieces, has been acquired. Explore additional resources to evaluate their functionality and review the report from the previous group to gain further insights. Divide the process into specific tasks, including using the camera to detect objects and identify chess pieces. Address challenges such as mitigating the effects of lighting and shadows to improve detection accuracy. Overcome the difficulty of recognizing pieces from a top-down angle, especially when the camera is positioned too far away or when attempting to capture two boards simultaneously. \\

The next steps involve reviewing previous work to determine how to effectively detect chess pieces and analyzing what others have done in similar projects. Work on the algorithm will begin, with plans to retrain models using a new dataset to improve accuracy. Understanding the entire process is key before adapting or retraining existing models. External resources such as Kaggle, for machine learning datasets, and Reddit discussions about digitizing chess games, was recommended by supervisor.