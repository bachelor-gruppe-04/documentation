\section{Meeting with product owner - March 21th}
\begin{tabular}{ll}
    \textbf{Date:} & Friday, March 21th 2025 \\
    \textbf{Time:} & 09.00 - 09.30\\
    \textbf{Location:} & A434 Ankeret, NTNU Ålesund \\
    \textbf{Participants:} & Birgitte Thoresen, Chris Sivert Sylte, Vegard Mytting and Arne Unneland\\
\end{tabular}

\vspace{0.5cm}

\subsection{Agenda}

\begin{itemize} 
    \item Run the application on localhost or on a website 
    \item How should the application handle illegal moves 
    \item Should previous games be saved 
    \item Feedback from product owner 
\end{itemize}

The application is intended for use during chess tournaments. The primary users are the spectators of the tournament. The camera is local and must be connected to a machine within the arena. The display should be visible onsite, but viewers should also be able to follow the game remotely from their own devices. Therefore, the application cannot be limited to running on localhost; it must be accessible as a website. This can be achieved via an API that sends the game data to the web interface. Since the viewers are the main users of the application, an admin view is not a high priority.

In manual chess games, where PGN notation is recorded by hand, illegal moves can occasionally occur. This might happen if a player is in check without anyone noticing or if a piece is moved illegally. If such moves go unnoticed, the game continues. When an illegal move is identified, there are various methods for handling it. In classical chess, players must trace back to the point of error using the PGN notation, usually with the help of a tournament judge. In Blitz games, this is more difficult, as PGN notation is typically not used and players may not remember previous moves. If an illegal move is discovered, the players must correct it—for example, if the white king is in check, the player must resolve the check by moving the king or blocking the attack.

On a digital board, if an illegal move is made, the board may stop registering further moves. This application should only display the chess game and should not prevent or interfere with gameplay, as online platforms do. Therefore, if an illegal move occurs, the application may freeze—essentially causing a crash. Since the application uses the Chess.js library (which enforces chess rules), an illegal move results in an error. The players should not be notified through the application, but the resulting PGN file becomes unusable. For a chess player, a PGN containing illegal moves is practically worthless for post-game analysis.

In the future, it may be useful to implement functionality to detect illegal moves or repeated positions. A repeated position occurs when the same board state appears three times, including the same player's turn. After three repetitions, a player can request a draw, and after five repetitions, the judge may declare a draw. For this purpose, a notification system within the application could help tournament judges make informed decisions.

The end result of the application should be a complete PGN file containing all game data, including time, location, and all moves made. Each file should correspond to one sequence, which could be a game or a round. A new PGN file should be generated for each new game. Additionally, a tournament file in .xml format will contain metadata such as player names, opponents, and match results. This file is generated using an older Windows application called Tournamentservice.

The product owner suggested that the first version of the application should assume a perfect scenario without illegal moves. Handling all possible errors adds complexity. The priority should be on detecting the correct piece positions and providing a clean visual display. Illegal moves can be ignored until the model is capable of accurately tracking a full game. The most important aspect is ensuring that valid moves are correctly detected.

For future improvements, Stockfish can be integrated to evaluate positions and show which player has the advantage. This can be containerized using Docker. Instead of requiring users to install Stockfish manually, the engine can be included within a Docker container. Users would only need to download the Docker package, with no additional installations required.

\subsection{Other} 
The product owner will be unavailable for the upcoming period but is reachable via email or messages.
