\section{Meeting with product owner - January 13th}
\begin{tabular}{ll}
    \textbf{Date:} & Monday, January 13th 2025 \\
    \textbf{Time:} & 08.30 - 09.30\\
    \textbf{Location:} & A433 Ankeret, NTNU Ålesund \\
    \textbf{Participants:} & Birgitte Thoresen, Chris Sivert Sylte, Vegard Mytting and Arne Unneland\\
\end{tabular}

\vspace{0.5cm}

First meeting with product owner, supervisor wanted to attend but was not available.\\

\textbf{What do you want?} \\
The product owner asked what suits us best. Chris Sivert is familiar with the chess club's tournaments. The chess club plays chess on traditional boards with physical pieces and clocks—not digital boards. On tournaments on TV, they use digital chessboards where chips are embedded in the pieces and the board, recording moves. The software communicates with the board and calculates moves based on chess rules (e.g., a rook moves across multiple files). 

PGN format is essential for recording games. Each square on the board has its own position. The system must recognize if the king has castled or if castling is not allowed. \\

\textbf{User Scenarios} 
Eliminate the need for digital boards by using a camera to capture a table or two and deduce the chess game to generate a PGN file.\\

There are two main systems: one for tournaments and one for live chess. The tournament system should identify matches based on the camera footage, predict players, and generate a PGN file. Display tournament info, including live games.\\

Cameras placed above the board may face detection challenges. Side-mounted cameras could simplify setup and focus on the board. They should detect valid moves, such as piece transformations (e.g., a pawn becoming a queen). \\

The focus should be on standard chess but accommodating additional types of chess would be interesting. One camera could potentially capture two boards. Clocks are secondary but could be considered later. Digital boards often have clocks integrated into the software. Error correction capabilities for moves (e.g., fixing misregistered moves) are important for ensuring perfect game records.\\

\textbf{Previous Work} 
ChessCam is a similar project. The goal is to find practical applications for industrial use. Pieces should be reset after every game, starting a new PGN file. Building on existing technologies to develop more complex systems is encouraged.\\


\textbf{Materials} 
Available equipment includes two Logitech 920 cameras that can be borrowed, along with chessboards and pieces. The club uses wooden boards with light and dark brown colors and off-white pieces. Materials can be picked up at the clubs local storage. \\


\textbf{Equipment and Setup} 
Affordable cameras are preferable since the club should be able to purchase many cameras. During tournaments, cameras could be connected to a PC in the secretariat. Modern setups could use smartphones as standalone servers, which are easier to scale.\\

Cameras could be placed on tripods at tables or secretariat desks. Streaming setups are used for tournaments, broadcasting on platforms like YouTube or Facebook. The club has software and gaming PCs, storing data on local disks or cloud services like Dropbox.\\

\textbf{Usage or Development of Models} 
Recognizing piece positions and board layout is crucial. Side-view cameras make identifying pieces easier, while overhead cameras are better for detecting board layout.\\

\textbf{Tips} 
Combine existing technologies to simplify development. Focus on integrating various components logically (e.g., tournament pairing and live displays). \\

Use a service like Stockfish for chess calculations to avoid developing rule-based move generation. A script could validate legal moves and generate PGN files. Include a UI displaying game details.\\

\textbf{Pieces} 
Ensure compatibility with various types of chess pieces, though most are relatively standard (e.g., Staunton design). It may be helpful to include both plastic and wooden pieces during development.\\

\textbf{Meetings} 
Meetings can be planned as needed. Spontaneous discussions are also possible. A meeting with the supervisor is not necessary, but will attend if wanted.