\section{Meeting with supervisor - January 10th}
\begin{tabular}{ll}
    \textbf{Date:} & Friday, January 10th 2025 \\
    \textbf{Time:} & 15.00 - 16.00\\
    \textbf{Location:} & A433 Ankeret, NTNU Ålesund \\
    \textbf{Participants:} & Birgitte Thoresen, Chris Sivert Sylte, Vegard Mytting and Saleh Abdel-Afou Alaliyat\\
\end{tabular}

\vspace{0.5cm}

The purpose of the initial meeting was to discuss questions with the supervisor and get started on the project. Some challenges identified early include the limited availability of images for machine learning. To address this, using pre-trained models could be beneficial. Training can be conducted either online or locally, depending on the setup. OpenStack is considered as a possible platform. Additionally, identifying the best methods for acquiring images and setting up cameras is crucial. Time should also be allocated for testing and obtaining feedback from stakeholders before the final delivery. \\

\textbf{How to Get Started?}  
The first step is to understand the problem presented by the product owner and find the best solution to address their needs. The initial email should be sent to establish contact with the group, schedule a meeting, and confirm availability. \\

\textbf{Questions for the Product Owner}  
Several questions need to be clarified with the product owner. It is important to determine their specific needs and expectations. For instance, do they already have a camera, or would they prefer a solution like a chess cam? Additionally, considerations should be made for typical chessboard colors (brown and white), challenges with lighting conditions, and ensuring compatibility with multiple types of chessboards and chess modes. \\

\textbf{Project Number}  
The project number is 04, and the group number is also 04. Arne is responsible for creating the groups. \\

\textbf{Report Template and Content}  
Both the NTNU LaTeX template and the required content must be followed. Using the template simplifies writing a bachelor report. For sprint and management, Git will be used since Jira is unavailable. \\

\textbf{Utilizing Available Resources}  
Existing resources should be leveraged, such as earlier bachelor’s theses, even though previous attempts were not successful. Mounting cameras from the ceiling could be an effective approach. Solutions involving Convolutional Neural Networks (CNNs) should also be investigated. Discussions with the product owner will provide further insights and recommendations. \\

\textbf{Scheduling Meetings}  
The initial meeting with the supervisor was conducted to clarify the project scope. A first meeting with the product owner, accompanied by the supervisor, is planned to present and understand the problem. Recurring meetings with the supervisor will be held every two weeks, tentatively on Friday mornings at 10:00 AM. Saleh will handle booking the meeting room starting from week 4. A meeting with the product owner should be arranged as soon as possible, supervisor is not available on Monday and Tuesday in week 3. 