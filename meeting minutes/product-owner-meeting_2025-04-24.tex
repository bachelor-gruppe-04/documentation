\section{Meeting with product owner - April 24th}
\begin{tabular}{ll}
    \textbf{Date:} & Thursday, April 24th 2025 \\
    \textbf{Time:} & 18.30 - 19.30\\
    \textbf{Location:} & Steinvågvegen 60, 6005 Aalesund \\
    \textbf{Participants:} & Birgitte Thoresen, Chris Sivert Sylte, Vegard Mytting and Arne Unneland\\
\end{tabular}

\vspace{0.5cm}

\subsection{Agenda}

\begin{itemize} 
    \item Review of work completed since the last meeting
    \item Feedback on color theme and layout
    \item Feedback from product owner 
\end{itemize}

The meeting began with a demonstration of the updated back-end application. The product owner tested the system, where chessboard moves were mapped using machine learning. The product owner made a white move followed by a black move, where the first move were correctly detected. Currently, moves are logged only after the subsequent move is made. During this demonstration, only the first move was logged correctly before the machine learning system failed to continue. After some waiting and additional attempts, the machine learning eventually began logging the chess moves accurately. It appears that the model needs a "warm-up" period before functioning reliably.

Next, the control panel was demonstrated. Here, the tournament organizer can select the number of cameras and reset one or all of the chessboards. The product owner suggested an improvement: when a camera detects the starting position, the board should automatically reset without requiring manual intervention through the control panel. 

Finally, the front-end application was demonstrated. The product owner was pleased with the chosen color scheme and overall design. For use on a large screen at a tournament, he requested that the tournament view display multiple boards simultaneously. For example, the view should show four boards, each displaying the players' names, clubs, ratings, and a preview of the board positions. Once a game is completed, it should be marked in the tournament view with the result of the game.

For further development, the group needs to integrate the back-end with the front-end. One development idea from the product owner was to create an application that starts with a simple switch. This would allow chess players to use the application without needing any technical knowledge. When the switch is turned on, the application would start, connect to the cameras, and begin recording games. Players could then play as usual, with the cameras passively recording the moves. When a chessboard returns to its starting position, the application would recognize that the game has ended and reset the board accordingly. After the tournament or session, players would simply turn off the switch to stop the application.

Another idea was to integrate Stockfish into the application, allowing players to analyze their games afterward. While the product owner shared several ideas for further development, he emphasized the importance of managing the remaining time carefully. The highest priority for this project is ensuring that the application can reliably and accurately read a chessboard.

The product owner also showed the group two of the digital chessboards used at the chess club, to compare them with the solution developed for this project. Te camera-based solution offers an extra benefit: it doubles as a security measure to monitor the equipment. The most significant difference is the cost of the equipment. This project requires only regular boards and pieces, along with a camera and a computer. In contrast, the digital boards require a specialized digital board (costing between 3,000 and 4,000 NOK even at a discounted price), special pieces (approximately 1,500 NOK), a special clock (around 1,000 NOK), and cables to connect everything to a computer. This setup typically costs over 5,500 NOK — and that’s for the most affordable option. 

\subsection{Other} 
Since the group will now focus on writing the report, this was the final scheduled meeting with the product owner. However, he will remain available via email for any questions related to the project.