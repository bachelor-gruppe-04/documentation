\chapter{Soft Requirements}

\begin{itemize}
    \item Commit messages should be written in lowercase, with proper nouns capitalized.
    \item Folder names should be written in lowercase.
    \item File names should be written in lowercase.
\end{itemize}

The group should actively use GitHub as a collaboration tool throughout the project. For every task, an issue should be created. Avoid creating trivial issues such as "fix typo" or similar. Each issue should have a descriptive name that clearly specifies what to do and where. For example: "Add a display for the PGN file." The issue description should comprehensively outline all tasks to be completed within that issue. Each issue should be assigned to a team member and labeled appropriately. Issues should be closed when they are completed, and any updates or changes that occur during the task should be documented as comments on the issue. \\

The group should use GitHub branches for development. Each new feature, improvement, or fix should have its own branch. Branches should be named according to the following convention: [type of branch; feature/improvement/etc.]/[name of issue it connects to]. Only merge branches that contain working and tested code. \\

When a feature is implemented, create a pull request (PR) for the branch. Add a well-documented description to the PR, linking the corresponding issue. Once the PR is approved and merged, the linked issue should automatically close. Request reviews from other team members, and at least one team member must approve the PR before merging it into the main branch. If improvements are needed, these should be commented on the PR for feedback. \\

Before merging a branch into the main branch, ensure that the latest main branch is first merged into the feature branch. This minimizes the risk of merge conflicts in the main branch. By following these guidelines, the team ensures efficient collaboration, clarity, and quality in the development process.\\