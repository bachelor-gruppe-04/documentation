\chapter{Soft Requirements}

\begin{itemize}
    \item Commit messages should be written in lowercase, with proper nouns capitalized.
    \item Folder names should be written in lowercase.
    \item File names should be written in lowercase.
\end{itemize}

\section{Github}
The group should actively use GitHub as a collaboration tool throughout the project. For every task, an issue should be created. Avoid creating trivial issues such as "fix typo" or similar. Each issue should have a descriptive name that clearly specifies what to do and where. For example: "Add a display for the PGN file." The issue description should comprehensively outline all tasks to be completed within that issue. Each issue should be assigned to a team member and labeled appropriately. Issues should be closed when they are completed, and any updates or changes that occur during the task should be documented as comments on the issue. \\

The group should use GitHub branches for development. Each new feature, improvement, or fix should have its own branch. Branches should be named according to the following convention: [type of branch; feature/improvement/etc.]/[name of issue it connects to]. This could be "feature/32-improve-method". Only merge branches that contain working and tested code. \\

When a feature is implemented, create a pull request (PR) for the branch. Add a well-documented description to the PR, linking the corresponding issue. Once the PR is approved and merged, the linked issue should automatically close. Request reviews from other team members, and at least one team member must approve the PR before merging it into the main branch. If improvements are needed, these should be commented on the PR for feedback. \\

Before merging a branch into the main branch, ensure that the latest main branch is first merged into the feature branch. This minimizes the risk of merge conflicts in the main branch. By following these guidelines, the team ensures efficient collaboration, clarity, and quality in the development process.\\

\section{Communication}
The internal group communicates through a shared communication channel, which is used for notifications regarding absences and sharing messages. This channel also serves as a space for sharing resources such as articles, videos, and other materials.\\

For communication with the supervisor and product owner, email is the primary method. The meeting leader is responsible for arranging meetings and preparing the meeting agenda. Emails should be addressed to the supervisor, the product owner, or both as necessary. Other team members should be copied on the email, but the meeting leader remains responsible for managing the communication process.

\section{Documentation}
Meeting minutes and the final report are prepared using \LaTeX{} and managed through Git. Agreements and other documents are stored on OneDrive. This approach is chosen because the templates used are time-consuming to integrate into \LaTeX{}. Since all documentation is finalized as PDF files, this method ensures efficiency and consistency.\\

All meeting minutes must follow a standardized style, as should the sprint retrospectives, to maintain uniformity throughout the documentation.

\section{Working Sessions and Meetings}

The group aims to maintain regular meetings, although deviations may occur. Until week 10, the group is concurrently taking the course INGA2300, which includes lectures on Mondays and Wednesdays, as well as group work. Consequently, less time is allocated to the bachelor thesis during the first 10 weeks. Despite this, the group has agreed to hold meetings every Thursday and Friday starting at 9:00. Adjustments to this schedule may occur as needed.\\

After the INGA2300 exam, the group will dedicate the entire week to working on the bachelor thesis, with daily work sessions starting at 9:00. The group intends to adhere to regular working hours but may work outside these hours if necessary. Each work session includes a "stand-up" meeting before lunch, approximately at 12:00, during which each team member updates the group on their progress and plans.\\

The group follows a two-week sprint cycle. Each sprint includes a sprint review, a sprint retrospective, and planning for the next sprint.

