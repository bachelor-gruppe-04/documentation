\section{Arbeidskontrakt for Birgitte Thoresem, Chris Sivert Sylte, Vegard Mytting}

\section*{Medlemmer}
Birgitte Thoresen, Chris Sivert Sylte, Vegard Mytting

\section*{Innledende tekst}
Denne arbeidskontrakten bygger på et sett med typiske mål, oppgavefordelinger, prosedyrer og retningslinjer for interaksjoner for studentarbeider. Arbeidskontrakten er utfylt med egne fortolkninger av hva man mener med disse og hvordan man skal oppnå dette.  
Det legges til eller fjernes punkter etter egen vurdering for tilpassing til oppgaven.

\section*{Roller og oppgavefordeling}
\textbf{Hvordan organiserer man arbeidet?}  
Hvilke roller/ansvarsområder er formålstjenlig for samarbeidet i prosjektgruppen?  

Eksempler på roller:  
\begin{itemize}
    \item Teamledelse
    \item Dokumentansvarlig
    \item Kvalitetssikring
\end{itemize}

Hva innebærer de ulike rollene, hvordan ivaretas de, hvem har ansvar for hva.

\section*{Prosedyrer (hvordan gjør man ting?)}
\begin{enumerate}[label=\Alph*.]
    \item \textbf{Møteinnkalling}  
    Når skal man ha møter. Hvordan innkalles det.

    \item \textbf{Varsling ved fravær eller andre hendelser}  
    Dersom man kommer for sent eller ikke kan møte.

    \item \textbf{Dokumenthåndtering}  
    Prosedyrer for lagring, samskriving, versjonshåndtering.

    \item \textbf{Innleveringer av gruppearbeider}  
    Ferdigstilling, kvalitetskontroll av innholdet, holde frister.
\end{enumerate}

\section*{Interaksjon (Hvordan opptrer man sammen?)}
\begin{enumerate}[label=\Alph*.]
    \item \textbf{Oppmøte og forberedelse}  
    Hva er godtatt som oppmøtetidspunkt til gruppemøter og forelesning. Hvilke krav har man til forberedelser.

    \item \textbf{Tilstedeværelse og engasjement}  
    Hva med bruk av PC til underholdning mens arbeid pågår.

    \item \textbf{Hvordan støtte hverandre}  
    Hva skal til for at man gleder seg til neste arbeidsdag.

    \item \textbf{Uenighet, avtalebrudd}  
    Hvordan håndteres uenighet, hvordan håndteres avtalebrudd (når ting ikke fungerer). Viktig å få fram hva dere som gruppe aksepterer av avvik.
\end{enumerate}

\end{document}
