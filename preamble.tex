\usepackage[nottoc]{tocbibind}
\usepackage[utf8]{inputenc}
\usepackage[T1]{fontenc}
\usepackage[Bjarne]{fncychap}
\usepackage{fancyhdr}
\usepackage[lmargin=1.5in, rmargin=1in, tmargin=1in, bmargin=1in]{geometry}
\usepackage{url}
\usepackage{listings}
\usepackage{amsmath,amssymb}
\usepackage{array}
\usepackage{multirow}
\usepackage{siunitx}
\usepackage[bf]{caption}
\usepackage{subcaption}
\usepackage{array, booktabs}
\usepackage{graphicx}
\usepackage{float}
\usepackage[export]{adjustbox}
\usepackage{subfig}
\usepackage{chngcntr}
\usepackage{color, xcolor}
\usepackage{comment}
\usepackage{afterpage}
\usepackage{siunitx}
\usepackage[backend = biber,
            style = ieee,
            date = long,
            urldate = iso8601,
            sorting = none,
            maxcitenames = 3,
            giveninits=false % Does NOT include initials
            ]{biblatex}
\usepackage{pgfplots}
\usepackage{pdfpages}
\usepackage{tabularx}
\usepackage{booktabs}
\usepackage{textcomp}
\usepackage{array}
\usepackage[colorlinks=true, allcolors=black]{hyperref}
\usepackage[toc, acronyms]{glossaries}

\usepgfplotslibrary{external}
\tikzexternalize

\sisetup{
  output-decimal-marker = {,},
  group-separator = {\,},
  group-minimum-digits = 4,
  table-number-alignment = center
}

\newcolumntype{L}[1]{>{\raggedright\arraybackslash}p{#1}}


%%%%%%%%%%%%%%%%%%%%%%%%% CONFIG %%%%%%%%%%%%%%%%%%%%%%%%%

% \renewbibmacro*{urldate}{
%     (Accessed:  \printfield{urlday}.\printfield{urlmonth}.\printfield{urlyear})
% }

\DeclareFieldFormat{urldate}{(Accessed: \thefield{urlday}.\addspace
  \mkbibmonth{\thefield{urlmonth}}\addspace
  \thefield{urlyear}\isdot)}

\setcounter{tocdepth}{2}
\setcounter{secnumdepth}{4}

% \counterwithin{figure}{section}
% \counterwithin{table}{section}

\bibliography{references}

\newcommand\MyBox[2]{
  \fbox{\lower0.75cm
    \vbox to 1.7cm{\vfil
      \hbox to 1.7cm{\hfil\parbox{1.4cm}{#1\\#2}\hfil}
      \vfil}%
  }%
}

\newcommand\myemptypage{
    \null
    \thispagestyle{empty}
    \addtocounter{page}{-1}
    \newpage
}

\newcommand{\saleh}[0]{Saleh Abdel-Afou Alaliyat}
\newcommand{\arne}[0]{Arne Unneland}
\newcommand{\peter}[0]{Peter Batchelor}
\newcommand{\birgitte}[0]{Birgitte Louise Vik Thoresen}
\newcommand{\chris}[0]{Chris Sivert Sylte}
\newcommand{\vegard}[0]{Vegard Arnesen Mytting}

%%%%%%%%%%%%%%%%%%%%%%%%% CODE %%%%%%%%%%%%%%%%%%%%%%%%%

\newfloat{code}{htbp}{loc}[section]

\floatname{code}{Code}

\newcommand{\loc}[0]{List of Code}

\newcommand{\listofcode}{\addcontentsline{toc}{chapter}{\loc}\listof{code}{\loc}}

\definecolor{background}{HTML}{FFFFFF}
\definecolor{text}{HTML}{000000}
\definecolor{keyword}{HTML}{0000FF}
\definecolor{string}{HTML}{A31515}
\definecolor{comment}{HTML}{008000}
\definecolor{number}{HTML}{098658}
\definecolor{function}{HTML}{795E26}

\lstset{
    backgroundcolor=\color{background},
    basicstyle=\ttfamily\color{text},
    keywordstyle=\color{keyword},
    stringstyle=\color{string},
    commentstyle=\itshape\color{comment},
    numberstyle=\color{number},
    showstringspaces=false,
    breaklines=true,
    frame=single,
    rulecolor=\color{text},
    tabsize=4,
    captionpos=b,
    escapeinside={(*}{*)},
    morekeywords={function, if, else, return},
    numbers=left,
    numberstyle=\tiny\color{gray},
    numbersep=10pt,
    stepnumber=1,
    firstnumber=1
}

%%%%%%%%%%%%%%%%%%%%%%%%% GLS %%%%%%%%%%%%%%%%%%%%%%%%%

\makeglossaries

\newglossaryentry{check}{
    name={check}, 
    description={A direct threat to the king that the opponent can legally escape from}
}

\newglossaryentry{checkmate}{
    name={checkmate}, 
    description={A position where the king is under attack and cannot escape, ending the game}
}

\newglossaryentry{stalemate}{
    name={stalemate}, 
    description={A situation where the player to move is not in check but has no legal moves; the game is declared a draw}
}

\newglossaryentry{capture}{
    name={capture}, 
    description={The removal of an opponent’s piece from the board as a result of a legal move}
}

\newglossaryentry{castling}{
    name={castling}, 
    description={A special move involving the king and one rook, where the king moves two squares toward the rook and the rook is placed on the square the king crossed}
}

\newglossaryentry{en-passant}{
    name={en passant}, 
    description={A special pawn capture that occurs when an opposing pawn moves two squares forward and lands adjacent; it may be captured as if it had moved only one square}
}

\newglossaryentry{promotion}{
    name={promotion}, 
    description={When a pawn reaches the opponent's back rank, it is converted into a queen, rook, bishop, or knight of the same color (excluding king or another pawn)}
}

\newglossaryentry{stockfish}{
    name={Stockfish}, 
    description={An open-source chess engine considered among the strongest in the world}
}

\newglossaryentry{lichess}{
    name={Lichess}, 
    description={A free, open-source online chess platform for playing, learning, and analyzing games}
}

\newglossaryentry{fide}{
    name={FIDE}, 
    description={The International Chess Federation; the global governing body of chess competitions}
}

\newglossaryentry{elo}{
    name={Elo rating}, 
    description={A system for calculating the relative skill levels of players in zero-sum games like chess}
}

\newglossaryentry{bullet}{
    name={bullet}, 
    description={A chess time control where each player has less than three minutes for the entire game}
}

\newglossaryentry{blitz}{
    name={blitz}, 
    description={A fast-paced chess format with a base time of three to ten minutes per player}
}

\newglossaryentry{rapid}{
    name={rapid}, 
    description={A chess format with a base time of more than ten minutes but less than 60 minutes per player}
}

\newglossaryentry{classical}{
    name={classical}, 
    description={A traditional chess format where each player has at least 60 minutes for the entire game}
}

\newglossaryentry{back-rank}{
    name={back rank}, 
    description={The first row (rank) on a player's side of the board at the start of the game, containing all major pieces}
}

\newglossaryentry{chess960}{
    name={Chess960}, 
    description={A chess variant, also called Fischer Random, where the starting position of the back-rank pieces is randomized}
}

\newglossaryentry{horde}{
    name={Horde}, 
    description={A chess variant in which white has 36 pawns and black plays with a standard set of pieces}
}

\newglossaryentry{racing-kings}{
    name={Racing Kings}, 
    description={A chess variant where players race to be the first to move their king to the eighth rank}
}

\newglossaryentry{git}{
    name={Git}, 
    description={A distributed version control system used to track changes in source code during software development}
}

\newglossaryentry{github}{
    name={GitHub}, 
    description={A web-based hosting service for Git repositories that supports collaboration, version control, and code review}
}

\newglossaryentry{repository}{
    name={repository},
    plural={repositories},
    description={A storage location for software code, documentation, and revision history}
}

\newglossaryentry{scrum}{
    name={Scrum}, 
    description={Scrum is a management framework that teams use to self-organize and work towards a common goal}
}

\newglossaryentry{agile}{
    name={Agile}, 
    description={A software development methodology emphasizing iterative delivery, adaptability, and close collaboration}
}

\newglossaryentry{latex}{
    name={LaTeX},
    description={A high-quality typesetting system used for producing technical and scientific documentation}
}

\newglossaryentry{python}{
    name={Python}, 
    description={A high-level, interpreted programming language known for its readability and broad applicability}
}

\newglossaryentry{wireframe}{
    name={wireframe},
    description={A simplified visual representation of a user interface layout, used for planning and structuring applications}
}


%%%%%%%%%%%%%%%%%%%%%%%%% ACR %%%%%%%%%%%%%%%%%%%%%%%%%

\newacronym{ai}{AI}{Artificial Intelligence}

\newacronym{ann}{ANN}{Artificial Neural Network}

\newacronym{api}{API}{Application Programming Interface}

\newacronym{cpu}{CPU}{Central Processing Unit}

\newacronym{cnn}{CNN}{Convolutional Neural Network}

\newacronym{dgt}{DGT}{Digital Game Technology}

\newacronym{fen}{FEN}{Forsyth-Edwards Notation}

\newacronym{http}{HTTP}{Hypertext Transfer Protocol}

\newacronym{ide}{IDE}{Integrated Development Environment}

\newacronym{i18n}{i18n}{internationalization}

\newacronym{yolo}{YOLO}{You Only Look Once}

\newacronym{leyolo}{LeYOLO}{Lightweight and Efficient You Only Look Once}

\newacronym{onnx}{ONNX}{Open Neural Network Exchange}

\newacronym{nms}{NMS}{Non Maximum Suppression}

\newacronym{llm}{LLM}{Large Language Model}

\newacronym{ml}{ML}{Machine Learning}

\newacronym{lstm}{LSTM}{Long Short-Term Memory}

\newacronym{ntnu}{NTNU}{Norwegian University of Science and Technology}

\newacronym{pgn}{PGN}{Portable Game Notation}

\newacronym{rest}{REST}{Representational State Transfer}

\newacronym{rfid}{RFID}{Radio-frequency Identification}

\newacronym{rnn}{RNN}{Recurring Neural Networks}

\newacronym{swc}{SWC}{Speedy Web Compiler}

\newacronym{tcp}{TCP}{Transmission Control Protocol}

\newacronym{tdd}{TDD}{Test-Driven Development}

\newacronym{ui}{UI}{User Interface}

\newacronym{uml}{UML}{Unified Modeling Language}

\newacronym{usb}{USB}{Universal Serial Bus}

\newacronym{ux}{UX}{User Experience}

\newacronym{vm}{VM}{Virtual Machine}

\newacronym{eco}{ECO}{Encyclopaedia of Chess Openings}

\newacronym{otb}{OTB}{over-the-board}

\newacronym{vscode}{VS Code}{Visual Studio Code}

\newacronym{wcag}{WCAG}{Web Content Accessibility Guidelines}

\newacronym{w3c}{W3C}{World Wide Web Consortium}

\newacronym{iou}{IoU}{Intersection over Union}

\newacronym{ap}{AP}{Average precision}

