\usepackage{tocloft} % For customizing the TOC
\usepackage{hyperref} % For clickable links in the TOC
\usepackage[a4paper, lmargin=1.5in, rmargin=1in, tmargin=1in, bmargin=1in]{geometry

% Adjusting TOC settings for alignment
\renewcommand{\cftchapleader}{\cftdotfill{\cftdotsep}}
\renewcommand{\cftdotsep}{2}


% Packages from NTNU Template
\usepackage[utf8]{inputenc} %to manage special characters
\usepackage[T1]{fontenc} %to manage special characters
\usepackage[Bjarne]{fncychap} %fancy chapter style (many more available, like Sonny or Lenny etc.)
\usepackage{fancyhdr} %to customize the headers
\usepackage[lmargin=1.5in, rmargin=1in, tmargin=1in, bmargin=1in]{geometry} %sets the margins for the pages
\setcounter{tocdepth}{2} %table of contents number depth for subsections (2 = x.x.x)
\setcounter{secnumdepth}{4} %numbering depth for headers for subsections in the text(4 = x.x.x.x)
\usepackage{url} %to include urls
\usepackage{listings} %include this if you want to include code in the thesis
\usepackage{amsmath,amssymb} %mathematical package
\usepackage{siunitx} %includes SI-units
\usepackage[bf]{caption} %makes float captions bold
\usepackage{array, booktabs} %to make better tables
\usepackage{graphicx} %to include graphics
\usepackage{float} %to include floats
\usepackage[export]{adjustbox} %to adjust floats
\usepackage{subfig} %to include subfigures
\usepackage{chngcntr} %will make it possible to change the counter for tables, figures etc. such as below
\counterwithin{figure}{section} %change counter for figures within sections (also possible to choose for each chapter
\counterwithin{table}{section} %change counter for tables within sections
\usepackage{color, xcolor} %edit e.g. text colors

\usepackage[backend = biber,
            style = ieee,
            date = long,     % Long: 24th Mar. 1997 | Short: 24/03/1997
            sorting = none,
            maxcitenames = 3,   % max names to include before et. al.
            ]{biblatex} %customize the look of your citations and bibliography
\bibliography{references}

\usepackage{comment} %to be able to comment out sections in the .tex files
\usepackage{afterpage} %to customize page commands such as below
\newcommand\myemptypage{
    \null
    \thispagestyle{empty}
    \addtocounter{page}{-1}
    \newpage
    } %sets new page command to insert an empty page without adding to the page counter or having a page number


\usepackage[english]{babel}
\usepackage{float}
\usepackage{graphicx}
\usepackage{titlesec}
\usepackage[toc, acronyms]{glossaries}
\usepackage{comment}
\usepackage[nottoc]{tocbibind}
\usepackage{subfig}
\usepackage[colorlinks=true, allcolors=black]{hyperref}

\makeglossaries

\newglossaryentry{git}{
    name={git},
    description={Et versjonskontrollsystem som sporer versjoner av filer}
}

\newacronym{ntnu}{NTNU}{Norwegian University of Science and Technology}