\usepackage[english]{babel}
\usepackage[nottoc]{tocbibind}
\usepackage[colorlinks=true, allcolors=black]{hyperref}

% Packages from NTNU Template
\usepackage[utf8]{inputenc} %to manage special characters
\usepackage[T1]{fontenc} %to manage special characters
\usepackage[Bjarne]{fncychap} %fancy chapter style (many more available, like Sonny or Lenny etc.)
\usepackage{fancyhdr} %to customize the headers
\usepackage[lmargin=1.5in, rmargin=1in, tmargin=1in, bmargin=1in]{geometry} %sets the margins for the pages
\usepackage{url} %to include urls
\usepackage{listings} %include this if you want to include code in the thesis
\usepackage{amsmath,amssymb} %mathematical package
\usepackage{siunitx} %includes SI-units
\usepackage[bf]{caption} %makes float captions bold
\usepackage{array, booktabs} %to make better tables
\usepackage{graphicx} %to include graphics
\usepackage{float} %to include floats
\usepackage[export]{adjustbox} %to adjust floats
\usepackage{subfig} %to include subfigures
\usepackage{chngcntr} %will make it possible to change the counter for tables, figures etc. such as below
\usepackage{color, xcolor} %edit e.g. text colors
\usepackage{comment} %to be able to comment out sections in the .tex files
\usepackage{afterpage} %to customize page commands such as below
\usepackage[toc, acronyms]{glossaries}
\usepackage[backend = biber,
            style = ieee,
            date = long,     % Long: 24th Mar. 1997 | Short: 24/03/1997
            sorting = none,
            maxcitenames = 3,   % max names to include before et. al.
            ]{biblatex} %customize the look of your citations and bibliography

%%%%%%%%%%%%%%%%%%%%%%%%% CONFIG %%%%%%%%%%%%%%%%%%%%%%%%%

\setcounter{tocdepth}{2} %table of contents number depth for subsections (2 = x.x.x)
\setcounter{secnumdepth}{4} %numbering depth for headers for subsections in the text(4 = x.x.x.x)

\counterwithin{figure}{section} %change counter for figures within sections (also possible to choose for each chapter
\counterwithin{table}{section} %change counter for tables within sections

\bibliography{references}

\newcommand\myemptypage{
    \null
    \thispagestyle{empty}
    \addtocounter{page}{-1}
    \newpage
    } %sets new page command to insert an empty page without adding to the page counter or having a page number



\newcommand{\aas}[0]{\textbf{Aalesunds Schaklag} }

\newcommand{\saleh}[0]{Saleh Abdel-
Afou Alaliyat}

\newcommand{\arne}[0]{Arne Unneland}

\newcommand{\peter}[0]{Peter Batchelor}

\newcommand{\birgitte}[0]{Birgitte Louise Vik Thoresen}

\newcommand{\chris}[0]{Chris Sivert Sylte}

\newcommand{\vegard}[0]{Vegard Arnesen Mytting}


%%%%%%%%%%%%%%%%%%%%%%%%% GLS %%%%%%%%%%%%%%%%%%%%%%%%%

\makeglossaries

\newglossaryentry{check}{
    name={check},
    description={An attack on the king, but unlike checkmate, this is one that your opponent \textit{can} escape}
}

\newglossaryentry{checkmate}{
    name={checkmate},
    description={An attack on the King that your opponent can’t escape}
}

\newglossaryentry{stalemate}{
    name={stalemate},
    description={The player to move isn’t in check, but they can’t move any of their pieces. It’s a draw}
}

\newglossaryentry{capture}{
    name={capture},
    description={Taking a piece from the board, so your opponent is a piece down}
}

\newglossaryentry{castling}{
    name={castling},
    description={The king and the rook swapping positions. The king moves two spaces from the starting position to the left or right, and the rook moves to next to it on the other side}
}

\newglossaryentry{en-passant}{
    name={en passant},
    description={A pawn that moves two squares forward can be taken by an opposing pawn that’s directly next to it on the following move}
}

\newglossaryentry{promotion}{
    name={promotion},
    description={A pawn that reaches the end of the board can become any piece you want (not to a king or another pawn)}
}

\newglossaryentry{stockfish}{
    name={stockfish},
    description={The strongest chess engine available to the public}
}

\newglossaryentry{git}{
    name={git},
    description={Et versjonskontrollsystem som sporer versjoner av filer}
}

\newglossaryentry{agile}{
    name={agile},
    description={Focus on delivering functionality quickly, responding to changing product specifications, and minimizing development overheads}
}

\newglossaryentry{python}{
    name={python},
    description={A programming language that lets you work quickly and integrate systems more effectively}
}

\newglossaryentry{leyolo}{
    name={LeYOLO},
    description={CNN based architecture for object detection}
}

%%%%%%%%%%%%%%%%%%%%%%%%% ACR %%%%%%%%%%%%%%%%%%%%%%%%%

\newacronym{pgn}{PGN}{Portable Game Notation}

\newacronym{fen}{FEN}{Forsyth-Edwards Notation}

\newacronym{ntnu}{NTNU}{Norwegian University of Science and Technology}

\newacronym{cnn}{CNN}{Convolutional Neural Network}

\newacronym{api}{API}{Application Programming Interface}

\newacronym{usb}{USB}{Universal Serial Bus}