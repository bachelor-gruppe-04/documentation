\usepackage[nottoc]{tocbibind}
\usepackage[utf8]{inputenc}
\usepackage[T1]{fontenc}
\usepackage[Bjarne]{fncychap}
\usepackage{fancyhdr}
\usepackage[lmargin=1.5in, rmargin=1in, tmargin=1in, bmargin=1in]{geometry}
\usepackage{url}
\usepackage{listings}
\usepackage{amsmath,amssymb}
\usepackage{array}
\usepackage{multirow}
\usepackage{siunitx}
\usepackage[bf]{caption}
\usepackage{subcaption}
\usepackage{array, booktabs}
\usepackage{graphicx}
\usepackage{float}
\usepackage[export]{adjustbox}
\usepackage{subfig}
\usepackage{chngcntr}
\usepackage{color, xcolor}
\usepackage{comment}
\usepackage{afterpage}
\usepackage[backend = biber,
            style = ieee,
            date = long,
            urldate = iso8601,
            sorting = none,
            maxcitenames = 3,
            giveninits=false % Does NOT include initials
            ]{biblatex}
\usepackage{pgfplots}
\usepackage{pdfpages}
\usepackage[colorlinks=true, allcolors=black]{hyperref}
\usepackage[toc, acronyms]{glossaries}

\usepgfplotslibrary{external}
\tikzexternalize


%%%%%%%%%%%%%%%%%%%%%%%%% CONFIG %%%%%%%%%%%%%%%%%%%%%%%%%

% \renewbibmacro*{urldate}{
%     (Accessed:  \printfield{urlday}.\printfield{urlmonth}.\printfield{urlyear})
% }

\DeclareFieldFormat{urldate}{(Accessed: \thefield{urlday}.\addspace
  \mkbibmonth{\thefield{urlmonth}}\addspace
  \thefield{urlyear}\isdot)}

\setcounter{tocdepth}{2}
\setcounter{secnumdepth}{4}

% \counterwithin{figure}{section}
% \counterwithin{table}{section}

\bibliography{references}

\newcommand\MyBox[2]{
  \fbox{\lower0.75cm
    \vbox to 1.7cm{\vfil
      \hbox to 1.7cm{\hfil\parbox{1.4cm}{#1\\#2}\hfil}
      \vfil}%
  }%
}

\newcommand\myemptypage{
    \null
    \thispagestyle{empty}
    \addtocounter{page}{-1}
    \newpage
}

\newcommand{\saleh}[0]{Saleh Abdel-Afou Alaliyat}
\newcommand{\arne}[0]{Arne Unneland}
\newcommand{\peter}[0]{Peter Batchelor}
\newcommand{\birgitte}[0]{Birgitte Louise Vik Thoresen}
\newcommand{\chris}[0]{Chris Sivert Sylte}
\newcommand{\vegard}[0]{Vegard Arnesen Mytting}

%%%%%%%%%%%%%%%%%%%%%%%%% CODE %%%%%%%%%%%%%%%%%%%%%%%%%

\newfloat{code}{htbp}{loc}[section]

\floatname{code}{Code}

\newcommand{\loc}[0]{List of Code}

\newcommand{\listofcode}{\addcontentsline{toc}{chapter}{\loc}\listof{code}{\loc}}

\definecolor{background}{HTML}{FFFFFF}
\definecolor{text}{HTML}{000000}
\definecolor{keyword}{HTML}{0000FF}
\definecolor{string}{HTML}{A31515}
\definecolor{comment}{HTML}{008000}
\definecolor{number}{HTML}{098658}
\definecolor{function}{HTML}{795E26}

\lstset{
    backgroundcolor=\color{background},
    basicstyle=\ttfamily\color{text},
    keywordstyle=\color{keyword},
    stringstyle=\color{string},
    commentstyle=\itshape\color{comment},
    numberstyle=\color{number},
    showstringspaces=false,
    breaklines=true,
    frame=single,
    rulecolor=\color{text},
    tabsize=4,
    captionpos=b,
    escapeinside={(*}{*)},
    morekeywords={function, if, else, return},
    numbers=left,
    numberstyle=\tiny\color{gray},
    numbersep=10pt,
    stepnumber=1,
    firstnumber=1
}

%%%%%%%%%%%%%%%%%%%%%%%%% GLS %%%%%%%%%%%%%%%%%%%%%%%%%

\makeglossaries

\newglossaryentry{check}{
    name={check}, 
    description={An attack on the king, but unlike checkmate, this is one that your opponent \textit{can} escape}
}

\newglossaryentry{checkmate}{
    name={checkmate}, 
    description={An attack on the King that your opponent can’t escape}
}

\newglossaryentry{stalemate}{
    name={stalemate}, 
    description={The player to move isn’t in check, but they can’t move any of their pieces. It’s a draw}
}

\newglossaryentry{capture}{
    name={capture}, 
    description={Taking a piece from the board, so your opponent is a piece down}
}

\newglossaryentry{castling}{
    name={castling}, 
    description={The king and the rook swapping positions. The king moves two spaces from the starting position to the left or right, and the rook moves to next to it on the other side}
}

\newglossaryentry{en-passant}{
    name={en passant}, 
    description={A pawn that moves two squares forward can be taken by an opposing pawn that’s directly next to it on the following move}
}

\newglossaryentry{promotion}{
    name={promotion}, 
    description={A pawn that reaches the end of the board can become any piece you want (not to a king or another pawn)}
}

\newglossaryentry{stockfish}{
    name={Stockfish}, 
    description={The strongest chess engine available to the public}
}

\newglossaryentry{git}{
    name={Git}, 
    description={A tool used for source code management}
}

\newglossaryentry{github}{
    name={GitHub}, 
    description={A cloud-based platform where one can store, share and work together with others to write code}
}

\newglossaryentry{repository}{
    name={repository},
    plural={repositories},
    description={A place where one can store code, files, and each file's revision history}
}

\newglossaryentry{agile}{
    name={agile}, 
    description={Focus on delivering functionality quickly, responding to changing product specifications, and minimizing development overheads}
}

\newglossaryentry{latex}{
    name={LaTeX},
    description={A document preparation system for high-quality typesetting}
}

\newglossaryentry{python}{
    name={Python}, 
    description={A programming language that lets you work quickly and integrate systems more effectively}
}

\newglossaryentry{wireframe}{
    name={wireframe},
    description={The basic layout and structure of an application}
}

%%%%%%%%%%%%%%%%%%%%%%%%% ACR %%%%%%%%%%%%%%%%%%%%%%%%%

\newacronym{ai}{AI}{Artificial Intelligence}

\newacronym{ann}{ANN}{Artificial Neural Network}

\newacronym{api}{API}{Application Programming Interface}

\newacronym{cpu}{CPU}{Central Processing Unit}

\newacronym{cnn}{CNN}{Convolutional Neural Network}

\newacronym{fen}{FEN}{Forsyth-Edwards Notation}

\newacronym{http}{HTTP}{Hypertext Transfer Protocol}

\newacronym{ide}{IDE}{Integrated Development Environment}

\newacronym{leyolo}{LeYOLO}{Lightweight and Efficient You Only Look Once}

\newacronym{llm}{LLM}{Large Language Model}

\newacronym{lstm}{LTSM}{Long Short-Term Memory}

\newacronym{ml}{ML}{Machine Learning}

\newacronym{ntnu}{NTNU}{Norwegian University of Science and Technology}

\newacronym{onnx}{ONNX}{Open Neural Network Exchange}

\newacronym{os}{OS}{Operating System}

\newacronym{pgn}{PGN}{Portable Game Notation}

\newacronym{rfid}{RFID}{Radio-frequency Identification}

\newacronym{rnn}{RNN}{Recurrent Neural Network}

\newacronym{tcp}{TCP}{Transmission Control Protocol}

\newacronym{tdd}{TDD}{Test-Driven Development}

\newacronym{ui}{UI}{User Interface}

\newacronym{uml}{UML}{Unified Modeling Language}

\newacronym{usb}{USB}{Universal Serial Bus}

\newacronym{ux}{UX}{User Experience}

\newacronym{vm}{VM}{Virtual Machine}

\newacronym{vscode}{VS Code}{Visual Studio Code}

\newacronym{wcag}{WCAG}{Web Content Accessibility Guidelines}

\newacronym{yolo}{YOLO}{You Only Look Once}

